% !TEX root = /Users/solveig/Documents/PhD/EEG/doc/eeg_main.tex
\documentclass[preprint,10pt,authoryear]{elsarticle}
\pdfoutput=1
\pdfminorversion=4
\usepackage[utf8]{inputenc}
\usepackage[T1]{fontenc}
\usepackage[english]{babel}
\usepackage{amsmath}
\usepackage{multicol}
\usepackage[usenames, dvipsnames]{xcolor}
\usepackage{graphicx}
\usepackage{caption}
\usepackage{subcaption}
\usepackage[margin=1.4in]{geometry}
\usepackage{float}
\usepackage{array}
\usepackage{graphicx}
\usepackage{siunitx}
\graphicspath{{../src/figures/}}
\usepackage{rotating}
\usepackage[labelfont={bf,up}, font={small}]{caption}
\usepackage{sidecap}
\sidecaptionvpos{figure}{c}
\usepackage[colorinlistoftodos]{todonotes}
\usepackage{adjustbox}
\usepackage{placeins}
\usepackage{lineno}
\usepackage{tabularx}
\DeclareUnicodeCharacter{00A0}{~}
\bibliographystyle{elsarticle-harv}
\biboptions{square}

\renewcommand\familydefault{\sfdefault}
\usepackage[scaled]{helvet}

\usepackage{units}


%\usepackage{soul}
\usepackage{soulutf8}
\usepackage{placeins}

\usepackage{nameref}
\usepackage[pdftex,breaklinks=true,colorlinks=true,linkcolor=blue,citecolor=blue,urlcolor=blue,filecolor=blue,pdffitwindow,pagebackref=false,bookmarks=true,bookmarksopen=true,bookmarksnumbered=true]{hyperref}
\usepackage[plain]{fancyref}
\usepackage{array}
\usepackage{multirow}

%custom color for \hlc
\newcommand{\hlb}[2][NavyBlue]{ {\sethlcolor{#1} \hl{#2}} }
\newcommand{\hlg}[2][Emerald]{ {\sethlcolor{#1} \hl{#2}} }
\newcommand{\hlp}[2][Purple]{ {\sethlcolor{#1} \hl{#2}} }


%boxes and highlight color for text updates, personified!
\newcommand{\snnote}[1]{\color{white}{\hlb{SN: #1 }}\color{black}}
\newcommand{\sntxt}[1]{{\color{NavyBlue}#1}}
\newcommand{\tvnnote}[1]{\color{white}{\hlg{TVN: #1 }}\color{black}}
\newcommand{\tvntxt}[1]{{\color{Emerald}#1}}
\newcommand{\gtenote}[1]{\color{white}{\hlp{GTE: #1 }}\color{black}}
\newcommand{\gtetxt}[1]{{\color{Purple}#1}}
\usepackage{ulem} % for deleting = strike out


\begin{document}
	
\begin{frontmatter}
	%\linenumbers
	
	\title{Biophysical modeling of ECoG, EEG and MEG signals\tvnnote{Alternativ tittel: 
			1) Towards a model-based understanding of the EEG signals.
			2) Towards accurate detailed biophysical modelling of EEG signals.
	}}
	
	\author{Solveig N\ae{}ss\corref{cor1}\fnref{label1}}
	\author{Geir Halnes\fnref{label2}}
	\author{Espen Hagen \fnref{label5}}
	\author{Eric Halgren\fnref{label3}}
	\author{Don Hagler\fnref{label3}}
	\author{Anders M. Dale\fnref{label4}}
	\author{Gaute T. Einevoll\fnref{label2,label5}}
	\author{Torbj\o{}rn V. Ness\fnref{label2}}
	
	\address[label1]{Department of Informatics, University of Oslo, Oslo, Norway}
	\address[label2]{Department of Mathematical Sciences and Technology, Norwegian University of Life Sciences, Ås, Norway}
	\address[label3]{Department of Radiology, University of California, San Diego, CA, USA}
	\address[label4]{Departments of Neurosciences and Radiology, University of California, San Diego, CA, USA}
	\address[label5]{Department of Physics, University of Oslo, Oslo, Norway}
	%\cortext[cor1]{correspondance: \href{}{}}
	
	%\date{\today}
	
	\begin{abstract}
		Many important brain signals, like the LFP, ECoG, EEG and MEG are thought to primarily reflect synaptic input to populations of geometrically aligned pyramidal neurons. Therefore, it is important to get a better understanding of the biophysics underlying the synaptic contribution to these brain signals. For a pyramidal neuron receiving a single synaptic input, it is well established that the ensuing extracellular potential will take the shape of a dipole in the far field limit, i.e., sufficiently far away from the neuron. This can potentially greatly simplify the link between the measured brain signals and the underlying neural sources, but this link has not yet been taken full advantage of in the framework of detailed biophysical forward modeling of brain signals. 
		Here we present a framework for reducing complex simulated neural activity to simple dipoles, and test the applicability of the approach. We find that the framework works excellently for calculating EEG signals, but not ECoG signals. We demonstrate how this approach can greatly simplify the link between the experimentally measurable EEG signal and the underlying neural sources, in a manner that is firmly grounded in the underlying biophysics. We demonstrate the power of this approach by showing that the EEG contribution from a neural population can be well represented by reducing it to a single dipole, based on the average obtained from the cells in the neural population. Such current dipoles can easily be used in both generic and complex head models, allowing for easy simulation and clean analysis of the neural origin of the EEG signal.
	\end{abstract}
	
\end{frontmatter}

%\linenumbers

	
%%%%%%%%%%%%%%%%%%%%%%%%%%%%%%%%%%%%%%%%%%%%%%%%%%%%%%%%%%%%%%%%%%%%%%%%
%%%%%%%%%%%%%%%%%%%%%%%%%%%%%%%INTRODUCTION%%%%%%%%%%%%%%%%%%%%%%%%%%%%%
%%%%%%%%%%%%%%%%%%%%%%%%%%%%%%%%%%%%%%%%%%%%%%%%%%%%%%%%%%%%%%%%%%%%%%%%

\begin{figure}[H]
	\centering
	\includegraphics[width=1.0\textwidth]{graph_abst.png}
	%\caption{\textbf{Graphical abstract}.
\tvnnote{NeuroImage har graphical abstracts, så det bør vi tenke på! Her er en kladd.}
%}
\label{fig:graphical_abstract}
\end{figure}



\section*{Higlights}
\tvnnote{NeuroImage vil ha Highlights med 3-5 bullet points}
\begin{itemize}
\item Arbitrary simulated neural activity can be automatically reduced to current dipoles
\item Such current dipoles can be used with generic or realistic head models for calculation of EEG or MEG signals
\item Allows easy simulation of EEG signals from both single cells and large-scale neural networks
\item Gives clean way of studying the neural origin of EEG signals
\end{itemize}


\section{Introduction}\label{sec:introduction}

\tvnnote{To include references to the relevant literature, we should read through BUZSAKI2012,SILVA2013, COHEN2017 and find appropriate papers to cite.
}

Electroencephalography (EEG) is the measurement of electrical potentials on the surface of the skull,
and such measurements have been an essential part of neuroscience for nearly a century \citep{SILVA2013}. Today, such measurements are still among the most important non-invasive methods (without surgical interventions) to study human cognitive functions, as well as to diagnose and study brain diseases \citep{COHEN2017, Pesaran2018}.

Despite EEG's long history and important position in neuroscience, we still know surprisingly little about the neural origin of the signal \citep{COHEN2017}. We have a relatively good understanding of the biophysical origin of the EEG signal: it mainly originates from large numbers synaptic inputs to cortical pyramidal cell populations \citep{NUNEZ2006, Pesaran2018}, however, we know very little about exactly what types of neural activity that cause even the most well-known characteristics of the EEG signal, such as different types of oscillations (e.g., alpha, beta, and gamma waves), as well as the shape of the signal in response to sensory stimuli (event-related potentials, ERPs) \citep{COHEN2017}. However, we know that these different characteristics of the EEG signal change in predictable ways as a result of brain disorders such as epilepsy or schizophrenia \citep{Light2013, MAKI2019} or as a result of the condition of the brain, such as sleep or attention \citep{Klimesch1998}. In other words, a better understanding of what aspects of the brain activity that is reflected in measured EEG signals could be of great help, both to understand what is going on in the brain and to cure brain disorders \citep{MAKI2019}.

The reason for our lack of understanding of the neural origin of the EEG signal is multifaceted,
but among the main causes are (i) the high number of neurons that contribute to the signal, and (ii)
strong ethical constraints on which tools can be used to investigate human brain. However, in recent years there have been major advances in several relevant branches of neuroscience that means a better understanding of the EEG signal may now be within reach \citep{COHEN2017}. On the experimental side, there has been a rapid development in technology and the methods used to study neural activity in animal studies, which in turn make important contributions our understanding of the human brain \citep{SILVA2013, COHEN2017, Pesaran2018, BRUYNS2017}, and on the modeling side there has been great development
in the tools needed \citep{HAGEN2018} and in available computing power \citep{EINEVOLL2019}.

Neural simulations has been important for understanding brain signals for a long time ...

However, EEG is slightly different.

Here we ...

\sntxt{	%Alt to 1st sentence below: When computing extracellular potentials in neural tissue, the standard approach is to apply a well-established compartment-based approach based on the underlying 		neural activity in the form of transmembrane currents.
	The neural activity underlying electric brain signals is predominantly thought of as transmembrane currents. Inevitably, these transmembrane currents form the basis of a well-established biophysical modeling scheme for computing extracellular potentials in neural tissue.
	For two reasons, however, translating neural activity into current dipoles, instead of currents, is necessary for studying the origin of extrabrainial electromagnetic signals: Firstly, the current dipole is the aim of inverse modeling of EEG and MEG signals \citep{NEYMOTIN2020}. Secondly, when modeling EEG signals, we depend on head models typically taking current dipole moments as input.
}

%%%%%%%%%%%%%%%%%%%%%%%%%%%%%%%%%%%%%%%%%%%%%%%%%%%%%%%%%%%%%%%%%%%%%%%%
%%%%%%%%%%%%%%%%%%%%%%%%%%%%%%%METHODS%%%%%%%%%%%%%%%%%%%%%%%%%%%%%%%%%%
%%%%%%%%%%%%%%%%%%%%%%%%%%%%%%%%%%%%%%%%%%%%%%%%%%%%%%%%%%%%%%%%%%%%%%%%
\section{Methods}\label{sec:methods}
%\tvnnote{I think it makes sense to more-or-less finish Results first, and then see how much we need to include here (some of it might fit into results, and figure captions. I like to have Methods in almost bullet-point format}
\tvnnote{Tror vi kan skrive litt mindre lærebok aktig, med mer "vi gjorde" enn "man kan"}
Neural activity generates electric currents in the brain, which in turn create electromagnetic fields. In this section, we explain how electric brain signals can be modeled in simple and more complex volume conductors.

\subsection{Forward modeling of electric potentials}
Due to negligibility of capacitive effects in the head, electric and magnetic signals effectively decouple, and we can apply the quasistatic approximation of Maxwell's equations for calculating these signals \citep{HAMALAINEN1993,NUNEZ2006}. In other words, for computing extracellular electric potentials, we envision the head as a 3D volume conductor, and combining Maxwell's equations with the current conservation law, we obtain the Poisson Equation for computing extracellular potentials \cite{GRIFFITHS1999}:


\begin{equation} \label{eq:poisson}
\mathbf{\nabla} \mathbf{J^p} = \mathbf{\nabla} (\sigma \mathbf{\nabla} \phi)~,
\end{equation}
where $\mathbf{J^p}$ is the primary current density, $\sigma$ is the extracellular conductivity and $\phi$ is the electric potential. The Poisson equation can be solved analytically for simple, symmetric head models, such as an infinitely big space and spherically symmetric models. For more complex head models, numerical methods such as the Finite Element Method (FEM) can be applied.

\subsubsection{Compartment-based approach}\label{subsubsec:multicomp}
Extracellular potentials generated by transmembrane currents can be calculated with a well-founded biophysical two-step forward-modeling scheme. The first step involves \textit{multicompartmental modeling} and incorporates the details of reconstructed neuron morphologies for calculating transmembrane currents \citep{STERRATT2011}. In the second step, Equation~\eqref{eq:poisson} is solved under the assumption that the extracellular medium is an infinitely large, linear, ohmic, isotropic, homogeneous and frequency-independent volume conductor. The transmembrane currents entering and escaping the extracellular medium can be seen as current sources and sinks, and give the extracellular potential $\phi$ at the electrode location $\mathbf{r}$
\begin{equation}
\phi(\mathbf{r}) = \frac{1}{4 \pi \sigma}\sum_{n=1}^N \frac{I_n}{|\mathbf{r} - \mathbf{r}_n|}~,
\label{eq:point_source}
\end{equation}
where $\mathbf{r}_n$ is the location of transmembrane current $I_n$, \sntxt{$N$ is the number of transmembrane currents} and $\sigma$ is the extracellular conductivity. %\tvnnote{Nevne N}.
This scheme is \sntxt{here} referred to as the \sntxt{\sout{multicompartment neuron model} compartment-based approach.} %\tvnnote{Mener vi "compartment based approach"?}
\sntxt{\sout{\sout{, and can easily be applied making use of} and was applied using the Python package \texttt{LFPy 2.0} running NEURON under the hood}} \citep{HAGEN2018,CARNEVALE2006}.
\tvnnote{Istedet for å si at det lett kan gjøres med LFPy, foreslår jeg å bare si vi brukte LFPy :-)}
\snnote{kanskje bare sløyfe siste setning, siden det uansett står under 2.3?}

\subsubsection{Current dipole approximation}\label{subsec:cda}
Analogous to how electric charges can create charge multipoles, a combination of current sinks and sources can set up \textit{current} multipoles \citep{NUNEZ2006} \tvnnote{Usikker på hvor mye mer kjent ladningsmultipoler er, så kan kanskje bare droppe første setning her?}.\snnote{Tanken var at ladningsmultipoler er et standard tema i elektromagnetismebøker, mens litteratur om strømmultipoler er ganske vanskelig å finne.. Men det er jo sikkert en del i målgruppa til denne artikkelen som ikke har hørt om ladningsmultipoler heller... Begge deler er jo nevnt i Nunez (men man må lete litt for å finne noe om strømmultipoler), så da kanskje det er ok å bare referere dit, og ikke si noe mer om ladningsmultipoler?} From electrodynamics, we know that extracellular potentials from a volume of current sinks and sources can be precisely described by expressing Equation~\ref{eq:point_source} as a multipole expansion \citep{NUNEZ2006}:
\begin{equation}\label{eq:multipole_expansion}
\phi(R) = \frac{C_{\text{monopole}}}{R} + \frac{C_{\text{dipole}}}{R^2} + \frac{C_{\text{quadrupole}}}{R^3} + ...~,
\end{equation}
when the distance R from the center of the volume to the measurement point is larger than the distance from the volume center to the most peripheral source \citep{JACKSON1998} \snnote{Sitere bokkapittel!}.
%For definitions of the monopole contribution $C_{\text{monopole}}$, dipole contribution $C_{\text{dipole}}$, quadrupole contribution $C_{\text{quadrupole}}$, see \cite{RIERA2012}.
%\tvnnote{Usikker på om vi bør bruke denne referansen, da litt av poenget i den artikkelen er at de ser monopoler? Har vi en annen? Se Gratiy, Pettersen, Einevoll, Dale (2013) Pitfalls in the interpretation of multielectrode data: on the infeasibility of the neuronal current-source monopoles. J Neurophysiol 109:1681–1682.} 
%\snnote{Aha!! Dette hadde jeg ikke fått med meg! Helt enig! Jeg klarer ikke å finne noen kilde som oppgir disse leddene for strømkilder (ikke ladninger). Går vel kanskje greit å ikke sitere noen her?} \tvnnote{Ja, kanskje bare evt legge inn sitering til bok-kapittelet vårt senere :-)}
%\snnote{smart!}
In neural tissue, the current monopole contribution is zero due to current conservation, since the transmembrane currents sum to zero at all points in time \citep{PETTERSEN2012}. Further, the quadrupole, octopole and higher order terms are negligible to the current dipole contribution when $R$ is sufficiently large.  Therefore, the extracellular potential from a neuron simulation can be estimated with the second term of the current multipole expansion; an approximation known as the \textit{current dipole approximation} \citep{PETTERSEN&EINEVOLL2008,PETTERSEN2014,NUNEZ2006}:
\begin{equation}\label{eq:dipole_approx}
\phi(\mathbf{r}) = \frac{C_{\text{dipole}}}{R^2} =\frac{1}{4 \pi \sigma} \frac{|\mathbf{p}| \cos \theta}{|{\bf r} - {\bf r}_p|^2}~.
\end{equation}
Here, $\mathbf{p}$ is the current dipole moment in a medium with conductivity $\sigma$, $R = |{\bf R}| = |{\bf r} - {\bf r}_p|$ is the distance between the current dipole moment at ${\bf r}_p$ and the electrode location ${\bf r}$, and $\theta$ denotes the angle between $\mathbf{p}$ and $\mathbf{R}$.
The current dipole moment ${\bf p}$ can be calculated from an axial current $I$ inside a neuron and the distance vector ${\bf d}$ traveled by the axial current: ${\bf p} = I{\bf d}$, analogous to a charge dipole moment. 
The current dipole approximation is applicable in the far-field limit, that is when $R$ is much larger than the dipole length $d = |{\bf d}|$, $R > 3d$ or $4d$ \citep{NUNEZ2006}.

%\snnote{I think I've been a bit back and forth regarding how exact the multi-dipole strategy is. Here's a summary after my final derivations: When inserting dipoles generated from a single current sink and a single current source into eq.4, you get no quadrupole,hexadecapol contribution, etc.. But there will be an octopole contribution (and a  $1/R^6$-contribtuion, $1/R^8$-contribution, etc). The multi-dipole approach is therefore not mathematically equivalent to the compartment-based approach (it is based on the current-dipole aproximation, after all). However, since the dipoles are tiny, the distance from the cell giving a neglibible octopole term is much closer here than for the big dipole length in the single-dipole approximation. I'm still not sure whether it would be better to call it the multi-dipole \textit{approach} or \textit{approximation}, since this is an approximation after all...}

\paragraph{Multi-dipole approach}\label{par:multi_dip}

From some multicompartmental neuron simulations (Figure \ref{fig:dipole_field}- \ref{fig:eeg_compare_cell_types}), we computed multiple current dipole moments, i.e., one for each axial current flowing between neighboring compartments in the neuron:
\begin{equation}
{\bf p}_k = I_k^{axial} {\bf d}_k.
\end{equation}
Here, $I_k^{axial}$ is an axial current traveling along distance vector ${\bf d}_k$, resulting in a current dipole moment ${\bf p}_k$.
%\tvntxt{\sout{Each of these current dipole moments contributes to the extracellular potential. }}
By inserting all the current dipole moments from a neuron simulation into the current dipole approximation (Equation~\ref{eq:dipole_approx}), we get a good estimate of the extracellular potential at any electrode location where the distance between the electrode and the nearest dipole is shorter than three or four times this dipole's length \citep{NUNEZ2006}.
Note that the length of each (multi-)dipole is equal to half the length of its corresponding neuronal compartment. %This means that each dipole is quite short (normally on the scale of a few tens of micrometers), meaning that the multi-dipole approach is normally a very good approximation at distances larger than 50-100 $\si{\mu m}$ from the neuron. \tvnnote{Tenker egentlig vi kan droppe siste setning} \snnote{Ja!}


%Note that these (multi-)dipoles are quite short (normally on the scale of a few tens of micrometers), meaning that the multi-dipole approach is normally a very good approximation at distances larger than 50-100 $\si{\mu m}$ from the neuron.
%\snnote{Maybe exclude the last sentence above? I only checked the segev 3a cell, and the largest dipole length was 12.5 micrometers, that's all it's based on.}
%\tvnnote{Jeg syntes den er lur å ha med, men kanskje koble det tydligere opp mot lengden på compartments i morfologiene?}
%\snnote{Noe sånt?}

\paragraph{Single-dipole approximation}\label{par:single_dip}
From each multicompartmental neuron simulation, we computed one single current dipole moment. This can either be done by summing up the multiple current dipole moments,

\begin{equation}\label{eq:dip_sum_axial}
\mathbf{p}(t) = \sum_{k=1}^{M} {\bf p}_k(t) = \sum_{i=1}^{M} I_k^{axial}(t) \mathbf{d}_k,
\end{equation}
where $M$ is the number of axial currents,
or equivalently from a position-weighted sum of all the transmembrane currents \citep{LINDEN2010, HAGEN2018}:

\begin{equation}\label{eq:dip_sum_trans}
\mathbf{p}(t) = \sum_{k=1}^{N} I_k^{trans}(t) \mathbf{r}_k,
\end{equation}
where N is the number of compartments in the multicompartmental neuron model and ${\bf r}_k$ is the position of transmembrane current $I_k^{trans}(t)$.
\tvnnote{Nenve hvordan man får dette ut i LFPy 2.0, type "We have implemented ..."}

%Combining the single current dipole moment with the current dipole approximation gives us the extracellular potential from a single neuron in the far-field limit ($R>3d$ or $4d$) \citep{NUNEZ2006}. \tvnnote{Trengs kanskje ikke?}




% Each of these current dipole moments will contribute to the extracellular potential from a neuron. And when assuming that our measurement points are further away from the dipole than the dipole length itself, we can plug the current dipole moments into the multipole expansion, and since the dipole contribution will be the only term contributing from a pure current dipole moment, the axtracellular potential 
%
%\begin{equation}
%\phi^{multi}_k(\mathbf{r}) =\frac{1}{4 \pi \sigma} \frac{|{\bf p}_k| \cos \theta}{|{\bf r} - {\bf r}_k |^2}~,
%\label{eq:dipole}
%\end{equation}
%
%where ${\bf r}_k$ is the location of current dipole moment ${\bf p}_k$. 
%



%The simplest possible neuron model from which we can calculate the current dipole moment is a  two-compartmental neural stick, where one compartment will act as a current sink and the other as a current source. The current dipole moment from this model, can be defined as the axial current from the current sink to the current source, times the distance between the sink and the source:
%
%\begin{eqnarray}
%{\bf p} = I^{axial}{\bf d}.
%\end{eqnarray}
%For this two-compartment model, $I^{axial} = I_{source} = I_{sink}$ and ${\bf d} = {\bf r}_{sink} - {\bf r}_{source}$
% Applying the point-source approximation and Kirchhoff's current law, we obtain the following relation:
%\begin{equation}\label{eq:dip_trans_to_axial}
%\mathbf{p} = I_{source} \mathbf{r}_{source} + I_{sink} \mathbf{r}_{sink}
%= I \mathbf{r}_{source} - I \mathbf{r}_{sink}
%= I (\mathbf{r}_{source} - \mathbf{r}_{sink}) = I \mathbf{d}~,
%\end{equation}
%where $\mathbf{d}$ denotes the distance between the two compartments, and $I$ is the axial current. Generalizing this relation to an N-compartment neuron model, we can calculate the current dipole moment by summing up multiple current dipole moments $\mathbf{p}_i^{multi}$, computed from axial currents $I_i^{axial}$ between neighboring compartments $i$ and $i+1$, separated by a distance $\mathbf{d}_i$:
%\begin{equation}\label{eq:dip_sum_axial}
%\mathbf{p}(t) = \sum_{i=1}^{N-1} I_i^{axial}(t) \mathbf{d}_i
%= \sum_{i=1}^{N-1} \mathbf{p}_i^{multi}(t)  ~.
%\end{equation}
%
%
%
%
%
%The current dipole moment can be calculated from transmembrane currents as follows:
%\begin{equation}\label{eq:dip_sum_trans}
%\mathbf{p}(t) = \sum_{n = 1}^{N} \mathbf{r}_n I_n(t)~,
%\end{equation}
%where $I_n$ is the transmembrane current in compartment $n$ at position $\mathbf{r}_n$.
%
%
%
%%Equations~\eqref{eq:dip_sum_trans}-\eqref{eq:dip_sum_axial} apply the point-source approximation. For proof of equality between the point-source and the line-source approximation when computing the current dipole moment from neuron simulation, see~\ref{sec:point_line_cdm}.


\subsection{Head models}
Electric potentials will be affected by the geometries and conductivities of the various parts of the head \citep{NUNEZ2006}, which is especially important for electrode locations outside of the brain. This can be incorporated into our extracellular potential calculations by applying simplified or more complex head models.

\subsubsection{Four-sphere model}
The four-sphere head model is a simple analytical model consisting of four concentric shells: brain tissue, cerebrospinal (CSF) fluid, skull and scalp, where the conductivity can be set individually for each shell, \citep{SRINIVASAN1998,NUNEZ2006}. The model solution is given in \cite{NAESS2017} and is found by solving the Poisson equation subject to boundary conditions ensuring continuity of current and electric potentials over the boundaries, as well as no current escaping the outer shell. This model is based on the current dipole approximation.

\subsubsection{New York Head model}\label{subsubsec:NYH}
The New York Head model is a realistic head model based on high-resolution, anatomical MRI-data from 152 adult heads \citep{HUANG2015}. The model was constructed by taking advantage of the reciprocity theorem, stating that the position of the electrode and the dipolar source can be switched without affecting the measured potential \citep{RUSH1969}. This means, that virtually injecting current at the locations of the EEG electrodes and using the finite element method \citep{LOGG2012} to compute the resulting potential anywhere in the brain, gives the link between current dipoles in the brain and the resulting EEG signals. This link was captured in a matrix known as the \textit{lead field} ${\bf L}$: 

\begin{equation}
{\bf L} = \frac{{\bf E}}{I}
\end{equation}
Here, $I$ is the injected current at the electrode locations and ${\bf E}$ is the resulting electric field in the brain. The lead field matrix gives us the precise link between a current dipole moment ${\bf p}$ in the brain and the resulting EEG signals ${\bf \Phi}$:
\begin{equation}
{\bf \Phi} = {\bf L} \cdot {\bf p}.
\end{equation}

We applied the New York Head model by dowloading the lead field ${\bf L}$ from \url{parralab.org/nyhead/}. The units incorporated in the lead field matrix was not immediately obvious, however from \cite{Dmochowski2017,HUANG2013} it appears that an injected current $I$ of 1 mA gives an electric potential $E$ in V/m, meaning that the lead field brings us from a current dipole moment ${\bf p}$ in the unit of mAm to EEG signals in the unit of V.


%The New York head model is an example of a numerical, complex head model based on 152 adult heads. In the New York Head Model, the link between current dipoles in the brain and resulting EEG signals was established using the finite element method \citep{LOGG2012,HUANG2015}. Even though developing such a model comes at a high financial and computational cost, applying the pre-solved model is quite straightforward: inserting a current dipole moment into an arbitrary brain location gives the resulting EEG signals less than a second.\tvnnote{Minutes? Ville trodd mye mindre. Nevne reciprocity?}\snnote{Obs! Denne hadde jeg glemt å fikse. 0.29s! } \snnote{Synes dette avsnittet er litt kjipt.. Enig i at reciprocity burde være med her. Burde vi isåfall fjerne det fra 3.5? Dette med hvor lang tid det tar å kjøre NYH har vi skrevet mer detaljert under 3.5.. Kunne evt. ha lagt første avsnitt fra 3.5 her og laget en kortere versjon til 3.5?}


\subsection{Simulation of neural activity}\label{subsec:simulation}
All neuron simulations were performed using the python package \texttt{LFPy}, running NEURON under the hood. 
For investigations of single-cell contributions to extracellular potentials, we applied three different morphologically reconstructed cell models: The human layer-23 pyramidal cell from \cite{EYAL2018}, the layer-5 pyramidal cell from rat cortex constructed by \cite{HAY2011} and a rat layer-5 chandelier cell; an interneuron model developed by \cite{MARKRAM2015}.

The pyramidal cell models were downloaded from \url{senselab.med.yale.edu/modeldb/}, see accession numbers 238347 \sntxt{(2013\_03\_06\_cell03\_789\_H41\_03)} and 139653 \sntxt{(cell1)} respectively, while we found the interneuron at the Neocortical Microcircuit Collaboration Portal (\url{bbp.epfl.ch/nmc-portal/microcircuit}) under layer-5, Chandelier Cell (ChC), continuous Non-accomodating (cNAC), \sntxt{(rp110201\_L\_idA\_-\_Scale\_x1.000\_y0.975\_z1.000\_-\_Clone\_3)}. %\snnote{There are several morphologies under each of the accession numbers. should we specify exactly which morphology we've used?}
%\tvnnote{Ja, gi navnet på morfologien}

For all simulations with passive ion channels only \sntxt{(Fig.~\ref{fig:dipole_field}-\ref{fig:eeg_compare_cell_types})}, 
%\tvnnote{Gjelder ikke hybrid modellen. Kanskje gi figurnummer?}
we used the following cell parameters: membrane resistance of 30000~$\Omega \text{cm}^2$, axial resistance of 150~$\Omega \text{cm}$ \citep{MAINEN1996} and a membrane capacitance of 1~$\mu\text{F}/\text{cm}^2$ %\tvntxt{\sout{(rounded up from}}
\citep{GENTET2000,STERRATT2011}. When active mechanisms were included in the simulations \sntxt{(Fig.~\ref{fig:ca_spike})}, all cell properties were incorporated as described in the specific cell's doumentation.

\sntxt{For results shown in Fig.~\ref{fig:dipole_field}-\ref{fig:ca_spike},} synaptic input was modeled as conductance-based, exponential synapses, (synapse type \texttt{Exp2Syn} in \texttt{NEURON}). The rise time constant was set to 1 ms and the decay time constant was 3 ms, synaptic reversal potential was 0 mV and the synaptic weight was set to 0.002, unless otherwise specified. %\tvnnote{Gjelder ikke hybrid modellen. Kanskje gi figurnummer?}

%\snnote{Burde denne vaert mer omfattende?:}
%\tvnnote{La på bittelitt, tenker det holder}
For modeling of population activity (Figure~\ref{fig:population}\sntxt{,~\ref{fig:compare_head_models}}), we  used the so-called hybrid scheme, and the simulation was unmodified from \cite{HAGEN2016}, except that all single-cell current dipole moments were recorded, and the EEG signals calculated.

\subsection{Code availability}
Simulation code to reproduce all figures in this paper is freely available from \url{https://github.com/solveignaess/EEG.git}

%%%%%%%%%%%%%%%%%%%%%%%%%%%%%%%%%%%%%%%%%%%%%%%%%%%%%%%%%%%%%%%%%%%%%%%%
%%%%%%%%%%%%%%%%%%%%%%%%%%%%%%%RESULTS%%%%%%%%%%%%%%%%%%%%%%%%%%%%%%%%%%
%%%%%%%%%%%%%%%%%%%%%%%%%%%%%%%%%%%%%%%%%%%%%%%%%%%%%%%%%%%%%%%%%%%%%%%%
\section{Results}\label{sec:results}
\normalsize

We introduce an approach for modeling electroencephalography (EEG) signals from detailed biophysical multicompartment cell models. For illustration, we first consider single synaptic input to single neurons in an infinite homogeneous head model, before moving on to a simple, generic head model. Finally, we study EEGs from population activity, also applying a realistic head model. Note that while we will only consider EEG signals, the general approach is equally valid for MEG signals.
%We introduce an approach for detailed biophysical modeling of ECoG, EEG and MEG signals by considering a single synaptic input to a biophysically detailed cell model. 
%
%Note that we will for the most part only consider EEG signals, but the general results are equally valid for MEG signals.

\subsection{At sufficiently large distances, extracellular potentials become dipolar}\label{subsec:cb_db_comp_inf}
%Excitatory synaptic input initiates a negative current at the synaptic input site (positive ions going into the cell), and due to current conservation \citep{NUNEZ2006} this negative current is exactly balanced by spatially distributed positive currents along the cellular membrane, as illustrated in Fig.~\ref{fig:dipole_field}\textbf{A} for a single apical excitatory synaptic input to a human cortical layer 2/3 pyramidal cell model \citep{EYAL2016}. In the standard approach to modelling extracellular potentials, here referred to as the {\it compartment-based approach}, the transmembrane currents of each of the cellular compartments correspond to a point current source/sink, but a mathematically equivalent formulation would be to instead consider the axial current of each cellular compartment as a small current dipole, which we will here refer to as the {\it multi-dipole approach} (Fig.~\ref{fig:dipole_field}\textbf{B}). By summing all these dipoles into a single dipole at a specific position, we obtain the {\it single-dipole approximation} (Fig.~\ref{fig:dipole_field}\textbf{C}).
%
%For illustration we first assume that the cell is positioned in an infinite homogeneous medium, in which case the extracellular potential can be easily calculated by the compartment-based approach (eq.~\ref{eq:point_source}) or from the multi-dipole or single-dipole approximation (eq.~\ref{eq:dipole}). Very close to the cell the extracellular potential will strongly depend on the exact distribution of transmembrane currents across the cellular morphology, and it will therefore typically not have the shape of a single dipole (Fig.~\ref{fig:dipole_field} \textbf{D,E} versus \textbf{F}). However, since the extracellular potential
%can be expressed as a multipole expansion (eq.~\ref{eq:dipole_expansion}) and all higher order terms will decay faster with distance than the dipole term, the extracellular potential becomes increasingly dipolar with distance from the cell \citep{LINDEN2010}, meaning that for the purpose of calculating extracellular potentials far away from the cell, the single-dipole approximation might be well justified (Fig.~\ref{fig:dipole_field}\textbf{G-I}).

%\sntxt{
%	
%In this section, we compare three strategies for computing extracellular potentials: the compartment-based approach, the multi-dipole approach and the single-dipole approximation.
%While the first is a well-established method for computing extracellular potentials in homogeneous volume conductors,
%%(in addition to studies such as \cite{NESS2016})
% inhomogeneous head models typically take current dipole moments as input, and must be based on approximations such as the latter two. In order to understand the differences between the three strategies, we here compare single-cell extracellular potentials modeled close to the cell, and the potential in the far-field limit.
%%\linebreak
%%The standard way of modeling extracellular potentials in homogeneous volume conductors is to apply the so-called compartment-based approach \ref{subsubsec:multicomp}. However, 
%\linebreak
%When computing extracellular potentials in neural tissue, well-established compartment-based approach based on transmembrane currents. Dipole-based approaches for modeling of extracellular potentials further away from the neuronal sources are important for two reasons: Firstly, 
%
%When it comes to inverse modeling of extracellular potentials further away from the neuronal sources, on the other hand, such as EEG and MEG, the goal is to find the neuronal sources in the form of current dipole moments; not transmembrane currents. In this section, we 
%
%(Section~\ref{subsubsec:multicomp}) in an infinite homogeneous medium is well jstified. When it comes to modeling electromegnetic signals further away from neuronal sources, like EEG and MEG, 
%
%Here, we investigate the extracellular potential of a single neuron in order to compare to dipole-based approaches with the 
%
%
%The standard forward-modeling scheme for extracellular potentials model extracellular potentials from neurons' transmembrane currents. When it comes to inverse modeling of EEG/MEG, on the other hand, the goal is to find the current dipole moments from the measured electromagnetic signals. Also, applying 
%}

%\tvnnote{Can we find a simple one-sentence motivation for why we might want to do this? Maybe start paragraph by breifly introducing why we want to go from compartment-based to dipole-based?}
%\snnote{Tuesday, we discussed that the overall motivation for going from compartment-based to dipole-based is 1) we need dipoles for head models and 2) inverse methods model dipoles, not currents. I wrote a short draft on this paragraph in the Introduction, so that we don't forget to spell this out there. Since the Results section should be able to stand alone, I guess it makes sense to mention this here as well. I found it very hard to make an easy-to-read one-sentence motivation here. Is it ok to skip part 2 of the motivation, in order to make this shorter, and assume that the reader goes through the introduction, if she wants the full story? I came up with a suggestion where I feel like the flow is ok, and that it leads up nicely to the rest of the section, but even though leaving out motivation part 2, I didn't manage to keep it within one sentence. Is this ok, or do you have any other suggestions?=)}

%We start by comparing three strategies for computing extracellular potentials: the compartment-based approach, the multi-dipole approach and the single-dipole approximation.
%While the first is a well-established method for computing extracellular potentials in homogeneous volume conductors,
%(in addition to studies such as \cite{NESS2016})
%inhomogeneous head models typically take current dipole moments as input, and must be based on approximations such as the latter two. In order to understand the differences between the three strategies, we here compare extracellular potentials generated by a single cell receiving excitatory synaptic input.

When modeling electric potentials within the brain, we can apply the well-established compartment-based approach assuming a homogeneous volume conductor (sec. \ref{subsubsec:multicomp}) \citep{EINEVOLL2013REVIEW}. However, this assumption is no longer valid when it comes to modeling EEG signals on the scalp, which calls for an inhomogeneous head model \citep{Ilmoniemi2019}. Such head models typically take current dipoles as input, as opposed to current sinks/sources, and must be based on the current dipole approximation \citep{NUNEZ2006}. Here, we introduce an approach for computing current dipoles from arbitrary simulated neural activity, and compare current-based and dipole-based modeling of electric potentials generated by a single cell receiving excitatory synaptic input.


%The neural activity underlying EEG signals is predominantly thought of as transmembrane currents. However, when it comes to inverse modeling of these signals, one estimates the origin of neural activity as current dipoles, instead of currents. Applying head models for computing EEG signals typically take 

Excitatory synaptic input initiates a negative current at the synapse location, since positive ions flow into the cell. Due to current conservation \citep{Koch1999}, this negative current is exactly balanced by spatially distributed positive currents along the cellular membrane, as illustrated in Fig.~\ref{fig:dipole_field}\textbf{A} for a single apical excitatory synaptic input to a human cortical layer-2/3 pyramidal cell model \citep{EYAL2016}.
%\sntxt{The simulation lasted 100 ms, with the synaptic input occurring after 20ms.}
%\tvnnote{Kan fjernes siden det er i figurtekst}
%\snnote{Maybe these terms will be defined in the methods section?}
%\tvnnote{I don't feel we have to introduce them, but maybe cite a basic textbook?}
In the standard procedure for modeling extracellular potentials, here referred to as the {\it compartment-based approach}, the transmembrane current in each cellular compartment corresponds to a point current source/sink. Another strategy is to consider the axial current of each cellular compartment as a small current dipole (see Equation~\eqref{eq:dip_sum_axial}), which we will refer to as the {\it multi-dipole approach} (Fig.~\ref{fig:dipole_field}\textbf{B}). By summing all these dipoles into one single dipole at a specific position, we obtain the {\it single-dipole approximation} (Fig.~\ref{fig:dipole_field}\textbf{C}).
For the sake of comparing these modeling approaches, we have assumed that the cell is positioned in an infinite homogeneous medium. Very close to the neuron, the extracellular potential will strongly depend on the exact distribution of transmembrane currents across the cellular morphology and  will, therefore, typically not take a purely dipolar shape 
%\tvnnote{be purely dipolar? It is dipole-like after all?}\snnote{agree!}
(Fig.~\ref{fig:dipole_field} \textbf{D,E} versus \textbf{F}). However, since the dipole contribution will dominate when we are further away from the current sources (see Equation~\ref{eq:multipole_expansion}), the extracellular potential becomes increasingly dipolar with distance from the cell \citep{LINDEN2010}, meaning that for the purpose of calculating extracellular potentials far away from the cell, the single-dipole approximation might be well justified (Fig.~\ref{fig:dipole_field}\textbf{G-I}). Note that there can be small differences between the results from the compartment-based and the multi-dipole approaches for electrode locations in the immediate vicinity of the current sources, due to the assumptions in the current dipole approximation (Fig.~\ref{fig:dipole_field}D versus Fig.~\ref{fig:dipole_field}E).%immediate vicinity of the neuron
%\tvnnote{Kanskje heller si "immediate vicinity of the current sources, due to the assumptions in the current dipole approximation" eller noe slikt? Tenker også vi kan droppe resten av setningen}
%\snnote{jepp!}
 %, however for the electrode locations considered in the rest of this paper, these differences will be negligible.


%For electrode locations in the immediate vicinity of the neuron, it is possible to see tiny differences between the compartment-based and the multi-dipole approaches, however

%\tvnnote{Here, I have used the terms compartment-based \textbf{approach}, multi-dipole \textbf{approach} and single-dipole \textbf{approximation}, but I'm not sure about that.}

%The applicability of the single-dipole model, as compared to the multicompartmental model, was tested in the simplest volume conductor possible, i.e. an infinite 3D volume conductor with constant conductivity. The extracellular medium conductivity was set to $0.3$~S/m \cite{HAMALAINEN1993,LOGOTHETIS2007,NUNEZ2006} and we used the Segev cell for the neuron simulation, as described in Section \ref{subsec:neuron_models}. A single excitatory synapse was placed on an apical dendrite, generating an exponential synaptic input current. Figure~\ref{fig:dipole_field} illustrates the difference in electric potential modeled with the multicompartmental, multi-dipole and single-dipole models, but also how the models give interchangeable results in the far-field limit.


\begin{figure}[H]
	\centering
	\includegraphics[width=1.0\textwidth]{fig_dipole_field_passiveTrue_single_syn481}
	\caption{\textbf{Extracellular potentials become dipolar in the far field limit}. 
		\textbf{A}: Layer 2/3 pyramidal cell from human \citep{EYAL2016} with an excitatory synapse placed on apical dendrite (red dot), and the resulting transmembrane currents for each compartment (blue and red arrows for negative and positive currents respectively).
		\textbf{B}: Green arrows represent the multiple current dipole moments between neighboring neural compartments.
		\textbf{C}: Gray arrow illustrates the total current dipole moment, that is, the vector sum of the dipoles in B.
	\textbf{D-F}: Extracellular potential in immediate proximity of the neuron, computed with the compartment-based approach, multi-dipole approach and single-dipole approximation, respectively. Note that the multi-dipole results differ slightly from the compartment-based approach when the distance from the measurement point to the nearest current dipole moment is short compared to the dipole length.
\textbf{G-I}: Same as D-F, but at a larger spatial scale (zoomed out). See $1$~mm scalebar in panel A, D and G.
}
\label{fig:dipole_field}
\end{figure}


\subsection{Single-dipole approximation is justified for EEG, but not ECoG signals}\label{subsec:cb_db_comp_4s}

%The ECoG and EEG signals are strongly affected by the very different conductivities of the CSF, skull and scalp \citep{NUNEZ2006}, and therefore, to test the applicability of the single-dipole approximation for computing ECoG and EEG signals, we used the four-sphere head-model \citep{NAESS2017, HAGEN2018, HAGEN2019}.
%
%For different positions of a single conductance-based excitatory synaptic input to the human cortical layer-2/3 pyramidal cell model \citep{EYAL2016}
%(Fig.~\ref{fig:compare_multi_single_dipole}\textbf{A}), we calculated the electric potential from the top of the cell to the top of the head, using both the multi-dipole approach and the single-dipole approximation (Fig.~\ref{fig:compare_multi_single_dipole}\textbf{B}). As expected we found that synaptic input to the distal apical dendrite gave larger potentials than synaptic input to the proximal apical dendrite due to the larger current dipole moment \citep{LINDEN2010, AHLFORS2015}, and that in both cases the potential was strongly attenuated by crossing the different layers of the head model, most strongly across the low-conducting skull (Fig.~\ref{fig:compare_multi_single_dipole}\textbf{B}). 
%
%We observed only small differences between the multi-dipole approach and the single-dipole approximation for the distal synaptic input, while for the proximal synaptic input we observed a substantial difference directly above the cell model, that was decreased along the path towards the top of the head (Fig.~\ref{fig:compare_multi_single_dipole}\textbf{B}). We quantified these differences by looking at the relative error, and found that for the distal synaptic input the relative error was 1.23\% and 0.12\% at the position of the ECoG and EEG electrodes respectively, while for the proximal synaptic input the relative error was 39.4\% and 3.36\% at the position of the ECoG and EEG electrodes respectively (Fig.~\ref{fig:compare_multi_single_dipole}\textbf{C})
%%\tvnnote{Solveig: Can you find exact numbers here?}. 
%
%In other words, the single-dipole approximation results in substantial errors at the position of the ECoG electrodes in both cases, but appears well-justified at the position of the EEG electrodes for some synaptic locations. In fact, we found that the relative error of the single-dipole approximation was strongly increased for synaptic locations $\sim$ 300~$\si{\um}$ above the soma (Fig.~\ref{fig:compare_multi_single_dipole}\textbf{D}). 
%This point can be considered a 'center-of-gravity' for the cells transmembrane currents  \citep{LINDEN2010, AHLFORS2015}, meaning that for a synaptic input at this location, equally much of the return current will be above and below the synaptic input, resulting in a very weak current dipole moments (Fig.~\ref{fig:compare_multi_single_dipole}\textbf{E}).
%This demonstrates that the relative error of the single-dipole approximation is negatively correlated with the amplitude of the current dipole moment (Fig.~\ref{fig:compare_multi_single_dipole}\textbf{F}), indicating that for large numbers of synaptic input, the total error of the single-dipole approximation can be expected to be small.

In order to test the applicability of the single-dipole approximation for calculating ECoG and EEG signals, we applied the four-sphere head model \citep{NAESS2017, HAGEN2018, HAGEN2019}.
Since the four-sphere head model takes current dipoles as input, the multi-dipole approach was used as benchmark: an assumption that should be well justified for the cell-to-electrode distances considered, see Section \ref{subsec:cb_db_comp_inf}.

%The ECoG and EEG signals are strongly affected by the very different conductivities of the CSF, skull and scalp \citep{NUNEZ2006}. To implement the different conductivities into our calculations, we applied the four-sphere head model when testing the applicability of the single-dipole approximation for computing ECoG and EEG signals \citep{NAESS2017, HAGEN2018, HAGEN2019}. \tvnnote{Korte ned avsnitt over til noe slikt som: "To account for the different conductivities of the different regions of the head, we used the four-sphere head model"?}
%Our virtual experimental set-up consisted of a vertical electrode array placed directly above a single neuron. The electrode extended from 100 $\mu$m above the top of the neuron and all the way to the scalp surface. For different positions of a single conductance-based excitatory synaptic input to the human cortical layer-2/3 pyramidal cell model \citep{EYAL2016} with passive ion channels only (Fig.~\ref{fig:compare_multi_single_dipole}\textbf{A}), we calculated the electric potential at the electrode measurement locations, using both the multi-dipole approach and the single-dipole approximation (Fig.~\ref{fig:compare_multi_single_dipole}\textbf{B}).

For different locations of a single excitatory synaptic input to a human cortical layer-2/3 pyramidal cell model \citep{EYAL2016}
(Fig.~\ref{fig:compare_multi_single_dipole}\textbf{A}), we calculated the electric potential at positions spanning from 100 $\mu$m above the top of the cell, to the surface of the head, using both the multi-dipole approach and the single-dipole approximation (Fig.~\ref{fig:compare_multi_single_dipole}\textbf{B}).
We used conductance-based synapses and included only passive membrane conductances, but we confirmed that instead using current-based synapses or a fully active cell model gave very similar results.  

The electric potential attenuated steeply with distance when crossing the different layers of the head model, most strongly across the low-conducting skull (Fig.~\ref{fig:compare_multi_single_dipole}\textbf{B}). 
For all synaptic input locations, we observed that the electric potential calculated with the single-dipole approximation markedly deviated from the multi-dipole approach directly above the cell model, but the difference strongly decreased with distance from the cell (Fig.~\ref{fig:compare_multi_single_dipole}\textbf{B}, full versus dashed lines for two chosen synapse locations). 
We quantified the model dissimilarities by looking at the relative error, and for a chosen distal synaptic input the relative error was 40.0\% and 1.06\% at the position of the ECoG and EEG electrodes respectively (Fig.~\ref{fig:compare_multi_single_dipole}\textbf{C}, green line). For a proximal synaptic input we observed a relative error of 76.1$\%$ at the ECoG position, and 7.61$\%$ at the EEG position (Fig.~\ref{fig:compare_multi_single_dipole}\textbf{C}, purple line).
In other words, the single-dipole approximation can result in substantial errors at the position of the ECoG electrodes, but appears justified at the position of the EEG electrodes for some synaptic locations. We found that calculating EEG signals with the single-dipole approximation gave relative errors peaking for synaptic locations $\sim$ 60 and 400~$\si{\um}$ above the soma (Fig.~\ref{fig:compare_multi_single_dipole}\textbf{D}),
but note that these synaptic input locations also gave relatively weak EEG signals (Fig.~\ref{fig:compare_multi_single_dipole}\textbf{E}).
This demonstrates that the relative error of the single-dipole approximation is negatively correlated with the amplitude of the scalp potential (Fig.~\ref{fig:compare_multi_single_dipole}\textbf{F}).
This is as expected, given that the strongest EEG signals are expected to be caused by dipole-like source/sink distributions (sec. \ref{subsec:cda}).
 
%\snnote{Jeg er litt usikker på de siste setningene her:}
%This demonstrates that the relative error of the single-dipole approximation is negatively correlated with the amplitude of the current dipole moment (Fig.~\ref{fig:compare_multi_single_dipole}\textbf{F}), indicating that for large numbers of synaptic input, the total error of the single-dipole approximation can be expected to be small. This is expected, given that purely dipolar-like current source distributions gives larger EEG contributions than multi-polar current source distributions.
%\snnote{Her antar vi at mye synaptisk input gir større dipol enn lite synaptisk input.. Det stemmer nødvendigvis ikke hvis man sammenligner én apikal synapse med mange synapser som er fordelt over hele morfologien... Jeg tror jeg skjønner hva du vil fram til, men klarer ikke å finne noen bedre måte å skrive det på.=S}

%As expected, we found that synaptic input to the distal apical dendrite gave larger potentials than synaptic input to the proximal dendrite, since the former leads to stronger current dipole moments \citep{LINDEN2010, AHLFORS2015}. For both synaptic input areas, the potential attenuated steeply with distance
%\tvnnote{maybe 'steeply with distance' so it does not sound like we are talking about time?} when crossing the different layers of the head model, most strongly across the low-conducting skull (Fig.~\ref{fig:compare_multi_single_dipole}\textbf{B}). 

%\sntxt{We observed differences between the multi-dipole approach and the single-dipole approximation for distal  and proximal synaptic input, see green and purple lines in Fig.~\ref{fig:compare_multi_single_dipole}\textbf{B}, respectively.} %For the proximal synapses, on the other hand, we found a substantial difference directly above the cell model.
%These difference did, however, decrease along the path towards the top of the head. %(Fig.~\ref{fig:compare_multi_single_dipole}\textbf{B}, purple lines).
%We quantified the model dissimilarities by looking at the relative error, and found that for the distal synaptic input the relative error was 40.0\% and 1.06\% at the position of the ECoG and EEG electrodes respectively. Proximal synaptic input gave a relative error of 75.3$\%$ at the ECoG position, and 7.32$\%$ at the EEG position, (Fig.~\ref{fig:compare_multi_single_dipole}\textbf{C}).
%\tvnnote{Solveig: Can you find exact numbers here?}. 
%In other words, the single-dipole approximation can result in substantial errors at the position of the ECoG electrodes, but appears well-justified at the position of the EEG electrodes for some synaptic locations. In fact, we found that the relative error of the single-dipole approximation \sntxt{peaked at synaptic locations $\sim$ 60 and 400~$\si{\um}$}
%\tvnnote{from the figure, it looks more like 100 um up to 300 um?} above the soma (Fig.~\ref{fig:compare_multi_single_dipole}\textbf{D}). 
%\sntxt{\sout{This point} These points} can be considered \sntxt{\sout{s}} 'center\sntxt{s}-of-gravity' for the cell's transmembrane currents \citep{LINDEN2010, AHLFORS2015}, meaning that a synaptic input at \sntxt{\sout{this location}these locations} will lead to equal amounts of current returning above and below the synapse, resulting in very weak current dipole moment\sntxt{s} (Fig.~\ref{fig:compare_multi_single_dipole}\textbf{E}).
%This demonstrates that the relative error of the single-dipole approximation is negatively correlated with the amplitude of the current dipole moment (Fig.~\ref{fig:compare_multi_single_dipole}\textbf{F}), indicating that for large numbers of synaptic input, the total error of the single-dipole approximation can be expected to be small.
%\snnote{Do you think the above sentences still are ok, even though the trend is not as clear anymore?}
%Note that similar results were found for simulations with current-based synapses, and when including active ion channels.


%We tested the applicability of the single-dipole model for computing ECoG and EEG signals, using the inhomogeneous four-sphere head-model. Since the four-sphere model takes current dipole moments as input, the multi-dipole model was used as ground truth. The Hay cell with conductance-based  exponential synaptic input was modeled with $30$ different input locations. Panel \textbf{D} of Figure~\ref{fig:compare_multi_single_dipole} show that apical synaptic input gives the largest current dipole moments. Moreover, panel \textbf{E} and \textbf{F} illustrate how relative errors for single-dipole model decay as function of dipole strength for ECoG and EEG signals. From panel \textbf{B} and \textbf{C} we can see how electric potential and relative error of single-dipole model fall as a function of distance from top of neuron to measurement electrode. The relative errors of the single-dipole model are higher for ECoG than for EEG signals. \snnote{give some example here!}
%\tvnnote{Did we also try current-based synapse? I think we should, so we can say it didn't matter.}
%\snnote{Did run. Didn't matter. Think we concluded that we don't need to mention this.}
%\snnote{Mention in methods.}
\begin{figure}[H]
	\centering
	\includegraphics[width=1.0\textwidth]{figure2_eeg.png}
	\caption{\textbf{Single-dipole approximation is justified for EEG but not ECoG signals}. 
		\textbf{A}: Illustration of four-sphere head model, where the pink, blue, green and purple spherical shells represent the brain, CSF, skull and scalp respectively. The pink inset shows the human layer-2/3 neuron \citep{EYAL2016}. $41$ simulations lasting 100 ms with a single synaptic input after 20 ms to cell with passive ion channels only, were performed for varying input locations, see colored dots. The z-component of the resulting current dipole moments for two synaptic input locations (large colored dots) are shown in inset below as functions of time. The data presented in this figure are computed at the simulation time points producing the largest current dipole moment for each synaptic input location.
		\textbf{B}: Magnitude of extracellular potential $|\phi|$ as function of distance from the top of the neuron, shown for two simulations with synaptic input locations marked by large colored dots in upper inset of A. In each simulation, we consider the time-point with the largest current dipole moment. Dashed lines show extracellular potentials computed with multi-dipole, and full lines show single-dipole calculations.
		%\tvnnote{Is the potential at a given timestep, or is it max? Should be mentioned here I think.}
		\textbf{C}: Relative error RE at EEG location comparing the single-dipole model to the multi-dipole model, as function of distance from top of neuron to measurement point.
		\textbf{D}: Relative error RE showing how single-dipole model deviates from multi-dipole model EEG calculations, as function of distance from soma to synapse location.
		\textbf{E}: Magnitude of EEG signal, $|\text{EEG}|$, as function of distance from soma to synaptic input location.
%		\tvnnote{Hadde det kanskje egentlig gitt mer mening å plotte EEG amplitude her? Kanskje mer ryddig å vise (enklere å forklare) at feilen i EEG-signalet er omvent korrelert med EEG amplitude?}
		\textbf{F}: Relative error, RE, showing how EEG calculations performed with the single-dipole approximation deviates from multi-dipole approach as a function of amplitude of the EEG signal, $|\text{EEG}|$.
%		\tvnnote{Kanskje du kan skrive navnet på de forskellige regionene med tilhørende farge over panel C eller noe slikt? }
		%	\tvnnote{Suggestion: How about running another simulation with all synapses active simulataneously, calculate the relative error, and add it to this panel (F) as a horizontal line? We would expect this to give a small error, helping to illustrate our point of the EEG being dominated by low-error dipoles.}
	}
	%	\tvnnote{Suggestion: How about running another simulation with all synapses active simulataneously, calculate the relative error, and add it to this panel (F) as a horizontal line? We would expect this to give a small error, helping to illustrate our point of the EEG being dominated by low-error dipoles.}
	\label{fig:compare_multi_single_dipole}
\end{figure}

\subsection{Compact description of EEG contribution simplifies analysis}\label{subsec:compact}
%In the previous section we showed that the single-dipole approximation was applicable for calculation of EEG signals from a single synaptic input to a human cortical Layer 2/3 cell model, and we now expand this to three different cell types (Fig.~\ref{fig:eeg_compare_cell_types}\textbf{A}), each receiving a large number of synaptic inputs arriving in waves with specific target regions on the cells (Fig.~\ref{fig:eeg_compare_cell_types}\textbf{B}).
In the previous section, we showed that the single-dipole approximation was applicable for calculation of EEG signals, and in this section we demonstrate that the single-dipole approximation can substantially simplify the analysis of the biophysical origin of EEG signals.

\subsubsection{Single-dipole approximation simplifies estimate of EEG contribution}

Pyramidal cells have a preferred orientation along the depth axis of cortex (here the $z$-axis), and the direction of the current dipole moment $\mathbf{p}$ can be expected to align with this axis since
radial symmetry will tend to make the orthogonal components ($\text{p}_x$, $\text{p}_y$) cancel \citep{HAGEN2018}. 
In contrast, interneurons show much less of a preferred orientation, and are therefore not expected to give any meaningful contribution to the EEG signal, except indirectly through input to pyramidal cells \citep{HAGEN2016}.
We illustrated this by applying the single-dipole approximation to three different cell types (Fig.~\ref{fig:eeg_compare_cell_types}\textbf{A}), each receiving a large number of synaptic inputs arriving in waves with specific target regions on the cells (Fig.~\ref{fig:eeg_compare_cell_types}\textbf{B}).

For the previously used human layer-2/3 cell (Fig.~\ref{fig:eeg_compare_cell_types}\textbf{A}, purple; \cite{EYAL2016}),
receiving a wave of excitatory synaptic inputs that were restricted to the uppermost 200~$\si{\um}$ of the cell (time=50~ms; Fig.~\ref{fig:eeg_compare_cell_types}\textbf{B}, purple dots), we observed a negative deviation of $\text{p}_z$ (Fig.~\ref{fig:eeg_compare_cell_types}\textbf{C}, purple line). For basal synaptic input (time=100~ms; Fig.~\ref{fig:eeg_compare_cell_types}\textbf{B}, purple line), the polarity of $\text{p}_z$ was instead positive, but of slightly lower amplitude than for apical input, as can be expected because the large area of the somatic region will cause strong return currents in the immediate vicinity of the synaptic inputs, and therefore an overall weaker current-dipole moment.


A uniform distribution of 400 synaptic inputs across the cell membrane with area-weighted probability (time=150~ms; Fig.~\ref{fig:eeg_compare_cell_types}\textbf{B}, purple line), only gave rise to small ripples in $\text{p}_z$, due to the substantial cancellation of current dipoles of opposite polarity. It is sometimes assumed that excitatory input is relatively uniformly distributed onto pyramidal cells, while inhibitory input is more directed to the perisomatic region \citep{Mazzoni2015, Telenczuk2019, Skaar2020, Telenczuk2020}. As expected, we found that the combination of the previously described wave of uniformly distributed excitatory synaptic input, combined with perisomatic inhibitory inputs gave rise to a clear negative response in $\text{p}_z$ (time=200~ms; Fig.~\ref{fig:eeg_compare_cell_types}\textbf{B}, purple line), which could be part of the explanation why inhibitory synaptic input in some cases have been found to dominate the LFP \citep{HAGEN2016, TELENCZUK2016}.

For a rat cortical layer-5 pyramidal cell model (Fig.~\ref{fig:eeg_compare_cell_types}\textbf{A}, blue; \cite{HAY2011}), the resulting current dipole moment was very similar in shape, but larger in amplitude, which was expected because the longer apical dendrite will tend to give larger current dipole moments (Fig.~\ref{fig:eeg_compare_cell_types}\textbf{C}, blue line).
Lastly, we used a rat cortical layer-5 interneuron model (Fig.~\ref{fig:eeg_compare_cell_types}\textbf{A}, green; \cite{MARKRAM2015}), but since the dendrites of interneurons are not structured into the same distinctive zones as pyramidal cells, the synaptic input caused very small net current dipole moments.

We calculated the EEG signals with the four-sphere head model, using both the multi-dipole (Fig.~\ref{fig:eeg_compare_cell_types}\textbf{D}, dotted lines) and the single-dipole (Fig.~\ref{fig:eeg_compare_cell_types}\textbf{D}, solid lines) approach. The single-dipole approach gave a maximum global error of  2.2$\%$, 3.5$\%$ and 0.34$\%$ for for the human layer-5 cell, the rat layer-5 cell and the rat interneuron, respectively.
%\sout{In all three cases, we found the single-dipole approximation to be in excellently agreement with the multi-dipole approach, demonstrating that}  
Importantly,
%\sntxt{The resulting EEG signals modeled with the four-sphere head model is shown in Fig.~\ref{fig:eeg_compare_cell_types}. We applied both the multi-dipole (dotted line) and the single dipole model (solid line).} \snnote{maximum global error is 1.5$\%$ for Segev, 2.7$\%$ for Hay and 49$\%$ for interneuron. Don't know if we should include these numbers. It seems like these maximum errors occur for apical input, while in fig. 2 the largest errors followed basal input..}
%Note that in all cases, 
the EEG signal is nearly fully described by the single-dipole $\text{p}_z$, that is, a single time-dependent array, which corresponds to a massive simplification in understanding the biophysical origin of the EEG signal, compared to considering the transmembrane currents and position of each cellular compartment. 


\begin{figure}[H]
	\centering
	\includegraphics[width=1.\textwidth]{figure3.png}
	\caption{\textbf{EEG signals and current dipole moment from three different cell types with various synaptic input}.
		\textbf{A}: The morphologies of a human L2/3 pyramidal cell (purple; \cite{EYAL2016}), a rat L5 pyramidal cell (blue; \cite{HAY2011}), and a rat L5 interneuron (green; \cite{MARKRAM2015}). The remaining panels display data connected to each cell type, see cell specific colors.
		\textbf{B}: Each dot represents an excitatory synaptic input at a specific time (x-axis) at a specific height of the neuron (z-axis, corresponding to panel A) for a specific cell type (color). The crosses mark inhibitory synaptic input. The four input bulks represent 1) 100 apical excitatory synapses, 2) 100 basal excitatory input, 3) 400 homogeneously spread-out excitatory synapses and 4) 400 homogeneously spread-out excitatory synapse and inhibitory basal synapses. The synaptic weights sum to 0.01 for all sets of excitatory/ inhibitory synapses in each wave. For the interneuron, which doesn't have typical "apical" or "basal" zones, the synapses were spread out all over the morphology for all input types.
		\textbf{C}: The x-, y- and z-components of the current dipole moment $\mathbf{p}$ for the three different cell types.
		\textbf{D}: EEG signals, $\phi$ from the three cell types computed with the four-sphere model.
%		\tvnnote{Nevne hvordan det er gjort for internevronet som ikke har apic og basal?}
%		\snnote{Har prøvd litt forskjellige løsninger her, men fant ut at det greieste var å ikke gjøre forskjell på apical/basal/uniform.}
	}
	\label{fig:eeg_compare_cell_types}
\end{figure}


\subsubsection{Current dipole moment expose dendritic calcium spikes}

%\cite{SUZUKI2017} recently demonstrated that dendritic calcium spikes could be recorded at the cortical surface with amplitudes of similar magnitude as the contribution from synaptic inputs. This demonstrates that active conductances may play an important role in shaping ECoG and EEG signals, and also suggests that information might be present in such signals which can be used in new ways to study the learning mechanisms associated with dendritic calcium spikes (cite). We found that the single-dipole approximation provides a convenient way of investigating the EEG-contribution of such dendritic calcium spikes.
%
%The previously introduced rat layer 5 cortical pyramidal cell model from \cite{HAY2011} can exhibit dendritic calcium spikes.
%When this cell model received a single excitatory synaptic input to the soma (Fig.~\ref{fig:ca_spike}\textbf{A}, blue dot) strong enough to elicit a somatic action potential (Fig.~\ref{fig:ca_spike}\textbf{B1}, blue), a small depolarization was also visible in the apical dendrite but no dendritic calcium spike was initiated (Fig.~\ref{fig:ca_spike}\textbf{B1}, orange). If instead the same somatic synaptic input was combined with an additional excitatory synaptic input to the apical dendrite 400~$\si{\um}$ away from the soma (Fig.~\ref{fig:ca_spike}\textbf{A}, orange dot), a dendritic calcium spike was initiated, which also induced two additional somatic spikes (Fig.~\ref{fig:ca_spike}\textbf{C1}).
%
%The simulated extracellular potential 30~$\si{\um}$ away from the soma had the shape of stereotypical extracellular action potentials in both cases, that is, a sharp negative peak followed by a broader and weaker positive peak (Fig.~\ref{fig:ca_spike}\textbf{B2, C2}), and the slow dendritic calcium spike was not reflected in the extracellular potential close to the soma (Fig.~\ref{fig:ca_spike}\textbf{C2}).
%We found that for the case with only a somatic spike and no calcium spike, the single-cell current dipole moment resembled the inverse of the extracellular potential (Fig.~\ref{fig:ca_spike}\textbf{B3}), while for the case with both somatic and dendritic spiking, a pronounced slow component was also present in the single-cell current dipole moment (Fig.~\ref{fig:ca_spike}\textbf{C3}).
%Note that somatic action potentials are typically not expected to contribute significantly to EEG signals (cite, but see \cite{TELENCZUK2015}), because the very short duration of spikes with both a positive and a negative phase implies that extreme synchrony is needed for spikes to sum constructively. Indeed, we found that when we calculated the sum of 1 000 instanced of the single-cell current dipole moment that was jittered (shifted) in time (normally distributed, standard deviation=10~ms), the case with the dendritic calcium spike has a $6.6$-fold larger maximum amplitude than the case with
%only the somatic spike (Fig.~\ref{fig:ca_spike} \textbf{B4} versus \textbf{C4}, 
%$\mathrm{max}|\mathbf{p}$| = 30.8~$\si{\uA}\cdot\si{\um}$ and 204.2~$\si{\uA}\cdot\si{\um}$ respectively). This demonstrates that dendritic calcium spikes are much more capable of summing constructively for a population of cells, and substantiates the role of dendritic calcium spikes in affecting ECoG/EEG/MEG recordings. 
%\tvnnote{Although our Ca-spikes appear quite different in shape to the ones demonstrated by Suzuki and Larkum (2017), who (in defence of our results) reported an 'unexpected shape'}
%
%It might initially seem surprising that the dendritic calcium spike is so strongly reflected in the single-cell current dipole moment, given that the transmembrane currents associated with the somatic action potential are much larger than those associated with the dendritic calcium spike: the maximum amplitude of the transmembrane currents of the somatic compartment was 45.1~$\si{\nA}$, compared to just 0.30~$\si{\nA}$ for the compartment with the apical synaptic input (Fig.~\ref{fig:ca_spike}A, blue and orange dots). However, the current dipole moment is given as the product between the amplitude of the current and the separation between the source and sink ($\mathbf{p}=I\mathbf{d}$; eq. \ref{eq:dip_trans_to_axial}), and while the currents associated with the somatic action potential will for the most part be contained within the somatic region, giving very small sink/source separations, the currents associated with the dendritic calcium spike will be distributed over a much larger part of the cell membrane (cite?).

\cite{SUZUKI2017} recently demonstrated that dendritic calcium spikes can be recorded at the cortical surface, and that the signal amplitudes can be similar to contributions from synaptic inputs.
%\snnote{I neste setning skjonner jeg ikke helt hva "which" refererer til: signals eller information? } \tvnnote{better now?}
%This demonstrates that active conductances may play an important role in shaping ECoG and EEG signals, and also suggests that information might be present in such signals which can be used in new ways to study the learning mechanisms associated with dendritic calcium spikes (cite). 
This demonstrates that active conductances may play an important role in shaping ECoG and EEG signals. Furthermore, it suggests that information about calcium dynamics might be present in such signals, and that this information could potentially be taken advantage of when studying learning mechanisms associated with dendritic calcium spikes \citep{SUZUKI2017}.
%\tvntxt{\sout{We found that the single-dipole approximation provides a convenient way of investigating the EEG-contribution \sntxt{\sout{of} from} such dendritic calcium spikes.}}

The previously introduced rat layer-5 cortical pyramidal cell model from \cite{HAY2011} can exhibit dendritic calcium spikes.
When this cell model received a single excitatory synaptic input to the soma (Fig.~\ref{fig:ca_spike}\textbf{A}, blue dot), strong enough to elicit a somatic action potential (Fig.~\ref{fig:ca_spike}\textbf{B1}, blue), a small depolarization was also visible in the apical dendrite (Fig.~\ref{fig:ca_spike}\textbf{B1}, orange). Even so, this did not initiate any dendritic calcium spike. However, when combining the same somatic synaptic input with an additional excitatory synaptic input to the apical dendrite, 400~$\si{\um}$ away from the soma (Fig.~\ref{fig:ca_spike}\textbf{A}, orange dot), we observed a dendritic calcium spike. The calcium spike did, in turn, induce two additional somatic spikes (Fig.~\ref{fig:ca_spike}\textbf{C1}).
For both synaptic input strategies described above, the extracellular potential simulated 30~$\si{\um}$ away from the soma took the shape of stereotypical extracellular action potentials: that is, a sharp negative peak followed by a broader and weaker positive peak (Fig.~\ref{fig:ca_spike}\textbf{B2, C2}). Further, we observed that the slow dendritic calcium spike was not reflected in the extracellular potential close to the soma (Fig.~\ref{fig:ca_spike}\textbf{C2}).
We found that for the case with only a somatic spike and no calcium spike, the single-cell current dipole moment resembled the inverse of the extracellular potential (Fig.~\ref{fig:ca_spike}\textbf{B3}), while for the case with both somatic and dendritic spiking, a pronounced slow component was also present in the single-cell current dipole moment (Fig.~\ref{fig:ca_spike}\textbf{C3}).
Note that somatic action potentials are typically not expected to contribute significantly to EEG signals (but see \cite{TELENCZUK2015}), because the very short duration of spikes with both a positive and a negative phase implies that extreme synchrony is needed for spikes to sum constructively. Indeed, we found that when we calculated the sum of 1000 instances of the single-cell current dipole moment that was jittered (shifted) in time (normally distributed, standard deviation=10~ms), the case with the dendritic calcium spike had a $6.6$-fold larger maximum amplitude than the case with
only the somatic spike (Fig.~\ref{fig:ca_spike} \textbf{B4} versus \textbf{C4}, 
$\mathrm{max}|\mathbf{p}$| = 30.8~$\si{\uA}\cdot\si{\um}$ and 204.2~$\si{\uA}\cdot\si{\um}$ respectively).
%\snnote{It is hard to read these numbers out from the plots.. Something odd with the axes here?}
This demonstrates that dendritic calcium spikes are much more capable of summing constructively for a population of cells, and substantiates the role of dendritic calcium spikes in affecting ECoG/EEG/MEG recordings. 
%\tvnnote{Although our Ca-spikes appear quite different in shape to the ones demonstrated by Suzuki and Larkum (2017), who (in defence of our results) reported an 'unexpected shape'}

It might initially seem surprising that the dendritic calcium spike is so strongly reflected in the single-cell current dipole moment, given that the transmembrane currents associated with the somatic action potential are much larger than those associated with the dendritic calcium spike: the maximum amplitude of the transmembrane currents of the somatic compartment was 45.1~$\si{\nA}$, compared to just 0.30~$\si{\nA}$ for the compartment with the apical synaptic input (Fig.~\ref{fig:ca_spike}{\bf A}, blue and orange dots). However, the current dipole moment is given as the product between the amplitude of the current and the separation between the source and sink ($\mathbf{p}=I\mathbf{d}$; Equation \ref{eq:dip_sum_axial}), and while the currents associated with the somatic action potential will for the most part be contained within the somatic region, giving very small sink/source separations, the currents associated with the dendritic calcium spike will be distributed over a much larger part of the cell membrane.

\begin{figure}[H]
	\centering
	\includegraphics[width=0.6\textwidth]{ca_spike_hay}
	\caption{\textbf{Current dipole moment expose dendritic calcium spikes.}
		\textbf{A}: Layer 5 cortical pyramidal cell model from rat \citep{HAY2011}, receiving either a single excitatory synaptic input to the soma evoking a single somatic action potential (blue dot, results in \textbf{B1-4}), or in addition an excitatory synaptic input to the apical dendrite, evoking a dendritic calcium spike and two additional somatic spikes (orange dot, results in \textbf{C1-4}). 
		\textbf{B1, C1}: Membrane potential at the two positions indicated in \textbf{A}.
		\textbf{B2, C2}: Extracellular potential 30~$\si{\um}$ away from the soma (red diamond in \textbf{A}), assuming for illustration an infinite homogeneous extracellular medium. 
		\textbf{B3, C3}: Single-cell current dipole moment. 
		\textbf{B4, C4}: Sum of 1000 instances of the single-cell current dipole moment (from \textbf{B3, C3}), that has been randomly shifted in time with a normally distributed shift with a standard deviation of 10~ms.
	}
	\label{fig:ca_spike}
\end{figure}

\subsection{Dipole approximation for populations of cells}\label{subsec:populations}
So far, we have only considered the EEG contributions from single cells, but real EEG signals are expected to reflect the activity of hundreds of thousands to millions of cells \citep{NUNEZ2006, COHEN2017}. 
%There are several ongoing large-scale modeling efforts that seek to simulate neural activity at different levels of biological detail. 
Biophysically detailed modeling of large populations is still in its infancy \citep{EINEVOLL2019} and at present typically include ``only'' a few tens of thousands of biophysically detailed cells \citep{MARKRAM2015, BILLEH2019}. Networks of point neurons on the other hand are regularly used to simulate hundreds of thousands \citep{BILLEH2019} or even millions of cells \citep{SENK2018, SCHMIDT2018}, however, point-neurons in their nature lack the ability to predict LFP, ECoG, EEG or MEG signals \citep{EINEVOLL2013REVIEW}.  

We therefore used the hybrid scheme \citep{HAGEN2016, SENK2018, Skaar2020}, where the network activity is first simulated in a highly computationally efficient manner with point neurons in NEST \citep{NEST} and the spiking activity of each neuron saved to file. Afterwards, each cell is modeled with biophysically detailed multicompartment morphologies, capable of predicting extracellular potentials, and the spikes of all the presynaptic neurons are used as synaptic input \citep{HAGEN2016, SENK2018}.

We used the large-scale point-neuron cortical microcircuit model from \cite{POTJANS2014, HAGEN2016}, which has $\sim$80 000 neurons divided into 8 different cortical populations, one excitatory and one inhibitory across four layers (L2/3 - L6), and can exhibit a diverse set of spiking dynamics including different oscillations and asynchroneous irregular network states \citep{HAGEN2016, BRUNEL2000}. 
The only difference from the original simulations by \cite{HAGEN2016} was the added calculation of current dipole moments.
We simulated transient thalamic synaptic input to layers 4 and 6 (Fig.~\ref{fig:population}\textbf{A}), and after the spikes had been mapped onto the multicompartment cell models (Fig.~\ref{fig:population}\textbf{B}), we calculated the LFP (Fig.~\ref{fig:population}\textbf{C}) similarly to \cite{HAGEN2016} (their Fig.~1), in addition to the current dipole moments of each cell.

For all cell populations, we found that the current dipole moment from individual cells could show large transient responses to thalamic input (Fig.~\ref{fig:population}\textbf{D}; gray lines show current dipole moment from individual cells in two example populations: L5 inhibitory (L5I) and L5 exitatory (L5E)), however, for all inhibitory populations the thalamic response was not visible in the average current dipole moment (Fig.~\ref{fig:population}\textbf{D}; black lines, L5I). The same was true for the current dipole moment components perpendicular to the depth axis for excitatory populations (Fig.~\ref{fig:population}\textbf{D}; L5E, p$_x$, p$_y$, black lines), but not for the component along the depth axis which had a substantial average response to the thalamic input (Fig.~\ref{fig:population}\textbf{D}; L5E, p$_z$, black line). 
These observations imply that, as previously noted, only the z-component of the current dipole moment from excitatory populations can be expected to contribute significantly to the EEG signal.

Our findings invite a simplified approach to calculate the EEG signal: Instead of calculating all single-cell EEG contributions and summing them (taking into account the position of the individual cells, similarly to what is done for the LFP signal), we can compute one summed p$_z$-component from each pyramidal cell population, place it in the population center, and calculate the resulting simplified EEG signal. This approximation is reasonable when the population radius is small compared to the distance from the population center to the EEG electrode. Note that the distance from the top of cortex to the top of the head is typically $\sim$10~mm, while the radius of the simulated population is here only $\sim$0.5~mm (Fig.~\ref{fig:population}; population outline in \textbf{B} is drawn in red in \textbf{E}).
Indeed, when we combined the current dipole moments with the four-sphere head model (Fig.~\ref{fig:population}\textbf{E}), we found that the full EEG signal that was calculated as the sum of the EEG contribution from each of the $\sim$80 000 cells at their respective positions, was in fact indistinguishable from the simplified EEG signal (Fig.~\ref{fig:population}\textbf{F}). This implies that a full understanding of the EEG signal from the simulated cortical activity can be extracted from a single time-dependent array for each pyramidal cell population (see Discussion for limitations).

We also compared the relative amplitude of the EEG signal from each population, and found that in this case, the excitatory population of L2/3 was the dominant source of the EEG signal (Fig.~\ref{fig:population}\textbf{F}), in line with (?)\tvnnote{(cite!)}. Note, however, that we expect this observation to be somewhat model-dependent, and that making strong claims about the contribution of different pyramidal cell populations would require a more thorough analysis than the scope of this study allows.

\begin{figure}[H]
	\centering
	\includegraphics[width=1.0\textwidth]{hybrid_with_EEG}
	\caption{\textbf{Large-scale neural simulations can be used to probe biophysical origin of EEG signals}. 
		\textbf{A}: Stimulus-evoked spiking activity from thalamic input (time $t= $ 900~ms, denoted by thin vertical line) in the cortical microcircuit model from \cite{POTJANS2014}. Dots indicate spike times of individual neurons, and populations are represented in different colors (I=inhibitory, E=excitatory).
		\textbf{B}: Multicompartment model neurons used to produce the measurable signals, with colors corresponding to panel \textbf{A}, showing one example morphology per population. Layer boundaries are marked at depths relative to cortical surface, z = 0. A laminar recording electrode with 16 contacts separated by
		100 $\mu$m (black dots) is positioned in the center of the population.
		\textbf{C}: LFPs calculated at depths corresponding to black dots in \textbf{B}.
		\textbf{D}: For the two L5 populations (L5I and L5E), the three components of the current dipole moment is shown for all individual cells (gray), together with the population average (black).
		\textbf{E}: Illustration of the four-sphere head model, with the red column corresponding the the outline of the population in panel \textbf{B}.
		\textbf{F}: The EEG signal from each population found by summing the single-cell EEG contribution of all individual cells (different colors), together with the total summed EEG signal (black). The simplified EEG signal was found by first summing the z-component of the current dipole moments for all pyramidal cells, that is L2/3E, L5E and L6E, and calculating the EEG from this single current dipole (red dashed).
	}
	\label{fig:population}
\end{figure}


\subsection{Dipole approximation in complex head models}
%The construction of such realistic head models are typically dependent on very expensive equipment, such as magnetic resonance imaging (MRI), to map the electrical conductivity of the entire head at resolutions of $\sim$0.5-1.0~mm$^3$ \citep{HUANG2015, HUANG2016}. Afterwards, numerical techniques such as the Finite Element Method (FEM) \citep{LOGG2012} can be used to calculate the signal at the EEG electrodes for arbitrary complex head geometries and arbitrary arrangements of current dipoles in the brain, but at a high computational cost.

Even though the four-sphere head model is convenient for generic EEG studies, many applications such as accurate EEG source analysis, require realistic head models \citep{DALE1999, VORWERK2014}.
%The four-sphere head model (Fig.~\ref{fig:compare_head_models}A) can be \sntxt{\sout{a}} convenient \sntxt{\sout{head model}} for some generic EEG studies, but for many applications more realistic head models will be needed \citep{DALE1999, VORWERK2014}. 
The construction of such realistic head models is typically dependent on expensive equipment, %\snnote{MRI er vel det essensielle "expensive equipment" her, saa ville ikke tenkt paa MRI som et eksempel, men heller hovedutstyret? Isaafall ville jeg skrevet "that is" isteden for "such as". Eller er det andre dyrt utstyr som ogsaa er viktig?}
%\tvnnote{du har nok rett}
that is magnetic resonance imaging (MRI), to map the electrical conductivity of the entire head at resolutions of $\sim$0.5-1.0~mm$^3$
%\snnote{are these numbers relevant, or is it better to just say: "very high resolutions"?}
%\tvnnote{I think they are good :-)}
\citep{HUANG2015, HUANG2016}. Afterwards, numerical techniques such as the Finite Element Method (FEM) \citep{LOGG2012} can be used to calculate the signal at the EEG electrodes for arbitrary arrangements of current dipoles in the brain, but at a high computational cost.
%\snnote{vet ikke om det var dette du mente i neste setning.. Ble det oppklarende eller misvisende, synes du?} \tvnnote{bra sann det er naa :-)}
The number of computing hours is, however, reduced by applying the reciprocity principle of Helmholtz. The reciprocity principle states, in short, that switching the location of a current source and a recording electrode will not affect the measured potential \citep{Malmivuo1995, Ziegler2014, HUANG2016, Dmochowski2017}. This implies that it suffices to use FEM to calculate the lead field in the brain from virtual current dipoles placed at each of the EEG electrodes. From the lead field matrix, we can infer the potential at the EEG electrodes, given an arbitrary arrangement of current dipoles in the brain.
Luckily, several such pre-solved complex head models are freely available, and one example is the {\it New York Head} (NYH) (Fig.~\ref{fig:compare_head_models}\textbf{A}), which we have applied here (\cite{HUANG2016}; \url{www.parralab.org/nyhead/}).

%\snnote{Det neste avsnittet har jeg sett meg litt blind paa. Blir veldig glad for endringer/ innspill her!til hvordan folgende del skal bli bedre!}
%\tvnnote{I think it could be improved by writing about the two positions at the same time: ``To illustrate the use of pre-solved complex head-models, we calculated and compared the EEG signals resulting from the current dipole moment obtained in Section subsec:populations inserted into either the four-sphere or the NY head model.
%We show results for the current dipole being positioned in two different brain regions of the NY head model, namely .... In both cases the current dipole moment was oriented along the depth axis of the cortex ... Note that because of the intrinsic complexity of the NY head model, the amplitude of the EEG signal is very different in the two cases, while it was much more similar for the four-sphere head model ...'' Maybe it is sufficient to give the distanced to the closest electrode only in the caption.}
%\snnote{Good idea!}

To illustrate the use of pre-solved complex head-models, we inserted the current dipole moment obtained in Section~\ref{subsec:populations} into the New York Head model (Fig.~\ref{fig:compare_head_models}\textbf{A}), at two manually chosen positions: one in the parietal lobe, and one in the occipital lobe, see stars in Fig.~\ref{fig:compare_head_models}\textbf{C, E}, respectively.
	In both cases, the current dipole moment was oriented along the normal vector of the brain surface, and the EEG signal was calculated.
	We then found dipole positions in the four-sphere head model (Fig.~\ref{fig:compare_head_models}\textbf{B}) with the same brain surface normal vectors and comparable distances to closest EEG electrode (Fig.~\ref{fig:compare_head_models}\textbf{D, F}, stars), and computed the EEG signal from the same current dipole moment. The computed EEG signals from the two head models were in this case relatively comparable in both spatial shape and amplitude (Fig.~\ref{fig:compare_head_models}\textbf{C-F}). The two head models also generated EEG signals of the same temporal shape, as expected, but while the four-sphere head model gave very similar EEG amplitudes for the two different dipole locations, the EEG amplitudes from the complex head model was much more variable, even for similar distances to the closest electrode (Fig.~\ref{fig:compare_head_models}{\bf G}, {\bf H}).

%\tvnnote{ikke sant? Har du avstandene?}
%\snnote{Ja! Parietal: 16.13mm i 4S og 16.76 i NYH.  Occipital: 16.90 mm i 4S og 14.64mm i NYH. Det som kanskje er litt dumt, er at der EEG-amplitudene for de to modellene ser mest sammenliknbare ut, dvs for dipol i occipital lobe er det 4S som har størst amplitude og lengst avstand fra dipol til nærmeste elektrode. Hvis avstanden hadde vært mer lik som i NYH ville altså ville vi hatt større forskjeller i amplitude for de to modellene.. Går det greit synes du, eller skal jeg gjøre et forsøk til på å flytte elektrodene i 4S-modellen, sånn at nærmeste elektrode kommer nærmere?}
The higher variability of the complex head model was also apparent in the decay of the maximum EEG amplitude with distance, which was perfectly smooth, exponential-like \citep{NUNEZ2006}, and very similar for the two locations in the four-sphere model, but very variable for the complex head model, although with the same general shape (Fig.~\ref{fig:compare_head_models}{\bf I}, {\bf J}).


%To illustrate the use of pre-solved complex head-models, we computed and compared EEG signals resulting from the current dipole moment obtained in Section~\ref{subsec:populations} inserted into the New York Head model (Fig.~\ref{fig:compare_head_models}\textbf{A}) and the four-sphere head model (Fig.~\ref{fig:compare_head_models}\textbf{B}). We show results for the current dipole being manually positioned in two different brain regions of the NYH model, namely the parietal lobe and the occipital lobe.

%In both cases the current dipole moment was oriented along the normal vector of the brain surface in the NYH model. Then, we found a comparable position in the four-sphere model, with the same brain surface normal vector \tvnnote{and comparable distance to closest EEG electrode?}. Further, we computed the resulting EEG signals applying the two head models as shown in Fig.~\ref{fig:compare_head_models}\textbf{C-F} for the time point of the population simulation giving with the largest current dipole moment amplitude ($t_{max}$). 

%\tvntxt{\sout{When looking at the EEG trace measured by the electrode closest to the dipole, we can see that t}T}he two \tvntxt{head} models generate \tvntxt{EEG} signals of the same shape, as expected (Fig.~\ref{fig:compare_head_models}{\bf G}, {\bf H}). Plotting the EEG signals at time point $t_{max}$ for all electrodes as a function of distance from current dipole moment to electrode, showed that the EEG signal decays exponentially with distance \tvnnote{Bortsett fra at den tilsynelatende begynner såvidt å øke igjen ganske fort (etter at den har krysset null?). Har du sjekket at det er exponentially} \citep{NUNEZ2006} (Fig.~\ref{fig:compare_head_models}{\bf I}, {\bf J}). Because of the intrinsic complexity of the NYH model, the amplitude of the EEG signal is very different for the two dipole positions. When applying the four-sphere model, on the other hand, the different dipole positions resulted in much more similar EEG amplitudes.

Note that despite the complexity the NYH model is substantially faster than our implementation of the four-sphere model. In order to simulate the EEGs from a dipole moment vector containing 1200 timesteps, the NYH model execution times were $<$0.4s, while the four-sphere model needed $<$1.5s.



%To illustrate the use of pre-solved complex head-models, we computed and compared EEG signals resulting from neural activity, by inserting the z-component of the population dipole described in Section~\ref{subsec:populations} into the four-sphere model and the New York Head model. We show results for the current dipole being positioned in two different brain regions of the N We started by choosing a dipole position on the top of a sulcus in the parietal lobe, and oriented the dipole as if the pyramidal cells in the neural population were aligned with the normal vector of the brain surface. Then, we found a\tvntxt{\sout{n equivalent} comparable} position in the four-sphere model, with the same brain surface normal vector. Further, we computed the resulting EEG signals applying the two head models as shown in Fig.~\ref{fig:compare_head_models}\textbf{C,D} for the time point of the population simulation giving with the largest current dipole moment amplitude ($t_{max}$). 
%
%When looking at the EEG trace measured by the electrode closest to the dipole, it appeared that the two models generate signals of the same shape, as expected. However, the amplitude of the EEG signal differ between the two models, even though the distance from source to electrode is quite similar (16.1~mm and 16.8~mm), and the conductivities of the different layers were set to be equal for both models. Since the New York Head model includes anisotropy and different thicknesses of the different layers, however, \tvntxt{differences are to be expected \sout{we expected no perfect overlap between the results}}.
%
%Plotting the EEG signals at time point $t_{max}$ for all electrodes as a function of distance from current dipole moment to electrode showed that the EEG signal decay exponentially with distance, as expected \citep{NUNEZ2006}. Due to folding of the cortex in addition to varying thickness of the different layers in the New York Head model, there is a larger variation in signal amplitude for a specific source-electrode distance for the NYH model compared to the four-sphere model.
%%In order to compare the New York Head and the four-sphere model, we plotted EEG traces computed at the location of the electrode with the shortest distance from the current dipole. The distance was 16.90~$\si{mm}$ in the four-sphere model and 14.64~$\si{mm}$ in the New York Head. In  Fig.~\ref{fig:compare_head_models}E we see how the EEG signals are similar in shape, but differ in amplitude. Since the New York Head model includes anisotropy and different thicknesses of the different layers, we expect no perfect overlap between the two models.
%%Plotting the EEG signals at time point $t_{max}$ as a function of distance from current dipole moment to electrode shows that the EEG signals decay exponentially as functions of, as expected \citep{Nunez2006}. Due to folding of the cortex in addition to larger variations in thickness of the different layers in the New York Head model, there is a larger variation in signal amplitude for a specific source-electrode distance for the NYH model compared to the four-sphere model.
%
%In the bottom row of Fig.~\ref{fig:compare_head_models} we have chosen a different dipole location: the top of a sulcus in the occipital lobe (see orange star in Fig.~\ref{fig:compare_head_models}\textbf{G,H}). The distance from dipole location to the closest EEG electrode was 16.90mm in the four-sphere model and 14.64mm in the New York Head. The resulting EEG signal amplitudes differ slightly between the two different models (Fig.~\ref{fig:compare_head_models}\textbf{I,J}), however, less than for the first dipole location.


%
%
%We started by placing the dipole Plotting EEG signals for the time point giving maximum population dipole amplitude as function of distance from dipole to electrode, shows that the EEG signals decay exponentially with distance from source (Fig.~\ref{fig:compare_head_models}D,H). Due to folding of the cortex in addition to larger variations in thickness of the different layers in the New York Head model, there is a larger variation in signal amplitude for a specific source-electrode distance for the NYH model compared to the four-sphere model. When looking at the EEG trace measured by the EEG electrode closest to the dipole, it appears that the amplitude of the EEG signal differs between the two models, even though the distance from source to electrode is quite similar (16.43 and 16.76mm), and the conductivities of the different layers are set to be equal in both models. Since the New York Head model includes anisotropy and different thicknesses of the different layers, we expected the results not to overlap perfectly.
%
%When inserting the population dipole described in Section~\ref{subsec:populations} into the four-sphere head model, we could easily estimate EEG signals at different electrode positions on the scalp surface (Fig.~\ref{fig:compare_head_models}C. The closest electrodes gave EEG traces of similar shapes, but various amplitudes due to differences in head layer thicknesses and shapes as shown in Fig.~\ref{fig:compare_head_models}C,D. Plotting EEG signals for time point giving maximum population dipole amplitude, as function of distance from dipole to electrode, shows that the EEG signals decay exponentially with distance from source. This has to do with angles, \citep{Nunez2006}, but I don't know if this should be mentioned.




\begin{figure}[H]
	\centering
	\includegraphics[width=1.0\textwidth]{figure6.png}
	\caption{\textbf{EEG signals from neuron population can be modeled with the four-sphere model and the New York Head}. EEG signals from population dipole resulting from waves of synaptic input to the cortical microcircuit model from \cite{POTJANS2014}. Population dipole was placed in two different locations: parietal lobe (C,D,G,I) and occipital lobe (E,F,H,J).
		{\bf A}: The New York Head model.
		{\bf B}: The four-sphere model consisting of four concentrical shells: brain tissue, CSF, skull and scalp. 
		{\bf C, D}: EEG signals ($\phi$) on scalp surface electrodes, seen from above, showing timepoint of the strongest current dipole moment $|{\bf p}|$ of the population simulation. Dipole is placed in the parietal lobe and location is marked by orange star, having the following coordinates in the NYH model: (55, -49, 57)~mm. EEG signals were computed with the New York Head model (C) and the four-sphere head model (D).
		{\bf E,F}: Equivalent to panel C,D, however, dipole is placed in occipital lobe, and electrodes are seen from the back of the head. Dipole coordinates for NYH model: (-24.3, -105.4, -1.2)~mm.
		{\bf G,H}: EEG trace computed with the four-sphere model (black) and the New York Head model (gray) on closest scalp surface electrode: i.e. the electrode with the shortest distance to the current dipole moment location. G shows results for dipole placed in parietal lobe (distance is 16.13~mm for the four-sphere model and 16.76~mm for NYH), while H shows results for dipole in occipital lobe (distance is 16.90~mm for the four-sphere model and 14.64~mm for NYH).
		{\bf I}: Absolute value of EEG signals from panel C, D generated by dipole in parietal lobe, plotted as function of distance from dipole to measurement electrode.
		{\bf H}: Equivalent to panel I, however, EEG signals from panel E, F generated by dipole in occipital lobe.
		%	\snnote{The layout of this figure was meant to be pedagogical, as the rows, columns show the same model/ dipole location, but it's probably not compact enough..}
		%	\snnote{The same number of electrodes (224) in the four-sphere model for both dipole locations. Can try to optimize this, to get minimal distance between dipole and closest electrode.}
		%	\snnote{compute angle between EEG electrode and current dipole moment/ current dipole position vector. This should also affect the amplitude..}
%		\tvnnote{Kanskje egentlig litt like farger i G-J? Hva med svart og grå?}
%		\snnote{God idé!}	
	}
	\label{fig:compare_head_models}
\end{figure}

%\tvnnote{When we have a figure with a single current dipole of, say 1 nA/um, and corresponding EEG potentials, of say, 1 uV, we should send it to the New York Head model people (Stefan Haufe), and ask for a confirmation that the magnitude of the EEG in relation to the current dipole makes sense}
%%%%%%%%%%%%%%%%%%%%%%%%%%%%%%%%%%%%%%%%%%%%%%%%%%%%%%%%%%%%%%%%%%%%%%%%
%%%%%%%%%%%%%%%%%%%%%%%%%%%%%DISCUSSION%%%%%%%%%%%%%%%%%%%%%%%%%%%%%%%%%
%%%%%%%%%%%%%%%%%%%%%%%%%%%%%%%%%%%%%%%%%%%%%%%%%%%%%%%%%%%%%%%%%%%%%%%%
\section{Discussion}\label{sec:discussion}
%Order:
%summary
%
%EEG is important but little is known, this work lays foundation
%Easy to study contribution, decoupled from head model (like Ih or Ca)
%Source localization
%
%Sinlge cells and head models fairly well understood but not circuit level
%Hard, but several ongoing projects, like Allen and HBP, which we expect will be increasingly important in neuroscience over the comming years.
%Our work enables EEG from large scale sims in complex head models
%
%TVB and HNN are doing important and related work, but at somewhat different levels
%
%Our approach can be used to calculate mappings from firing rates to EEGs through the kernel method, thereby meeting TVB.
%
%A better understanding of EEG signals could lead to important discoveries (Mäki-Marttunen et al. 2019)
%Our approach should ideally be used in combination with animal studies (Bruynes-Haylett et al. 2017)

%\subsection*{Summary}
In this paper we have introduced an approach for reducing arbitrary simulated neural activity to single current dipoles (Fig.~\ref{fig:dipole_field}), and implemented the approach into the open-source python tool LFPy 2.0 (note that exact implementation presented here has already been used by \cite{HAGEN2018} and \cite{MAKI2019}). We verified that the approach was applicable for calculating EEG, but not ECoG signals (Fig.~\ref{fig:compare_multi_single_dipole}-\ref{fig:eeg_compare_cell_types}), and showed examples of how this approach can be a powerful tool for investigating and understanding single-cell EEG contributions (Fig.~\ref{fig:eeg_compare_cell_types}-\ref{fig:ca_spike}). Furthermore, we demonstrated that the presented approach could be easily integrated with existing large-scale simulations of neural activity, and how it can be used to construct compact representations of the EEG contributions from entire neural populations, while still firmly grounded in the underlying biophysics (Fig.~\ref{fig:population}). Finally, we demonstrated how the simulated current dipoles, from single cells or large neural populations, can be direcly inserted into complex head models for calculating more realistic EEG signals (Fig.~\ref{fig:compare_head_models}).




%\subsection*{Importance of work}
EEG signals have been an important part of neuroscience for a century, but still very little is known about the neural origin of the signal \citep{COHEN2017}. This work lays some of the foundation for obtaining a better understanding of the EEG signal, by allowing easy calculation of EEG signals from arbitrary simulated neural activity.
The presented formalism is well suited for modelling studies of the contribution to the EEG signal from different potential neural origins, like different cell types, different ion channels, or different synaptic pathways. 
In particular, notice that to judge the contribution of, say, calcium spikes \citep{SUZUKI2017}, or I$_{\rm h}$ currents on human EEG signals \citep{NESS2016, NESS2018, KALMBACH2018}, one only needs to understand how the z-component of the resulting population current dipoles are effected, without needing the added complexity of head models. 
This decoupling of the current dipole moments and head models presents a simplified way of investigating the origin of the EEG signal.

EEG measurements are often used for source localization, which aims to identify the underlying cortical current dipoles \citep{NUNEZ2006, Gramfort2014, Ilmoniemi2019}. However, such reconstructed current dipoles are generic in the sense that they are typically not intended to represent specific neural populations. By allowing for calculation of current dipoles from cortical populations, this work takes a step towards consolidating the so far mostly separate scientific disciplines of neural modelling and EEG data analysis (but see also \cite{NEYMOTIN2020}).

%\subsection*{Limitations}
Arguably, we today have a reasonably good understanding of how single neurons operate, that is, how they respond to synaptic input, and how multitudes of such synaptic inputs combine to produce action potentials \citep{EINEVOLL2019}. Similarly, we have a reasonably good understanding of the measurement physics for the EEG signal, that is, how the neural currents affect the electric potentials recorded outside the head \citep{NUNEZ2006, COHEN2017}.
The challenge in understanding EEG signals is therefore closely related to the greatest challenge in modern neuroscience: understanding neural networks. Understanding such complicated dynamical systems typically requires some kind of computational modelling \citep{EINEVOLL2019}, however, the complexity of neurons, and the complexity and size of the neural networks involved in even the simplests of cognitive tasks makes this a daunting challenge. The steady increase in available computing power, in combination with the ever increasing knowledge on synptic connectivity patterns is however making this approach increasingly attractive \citep{Reimann2013, Egger2014, MARKRAM2015, HAGEN2016, Gratiy2018, Arkhipov2018, Reimann2019, BILLEH2019}, and several ongoing research projects are pursuing such modelling efforts, like the Allen Institute for Brain Science and the Human Brain Project. While such biophysically detailed large-scale neural simulation is still in its infancy, we expect it to become an increasingly important research tool in neuroscience \citep{EINEVOLL2019}, and by enabling simulation of EEGs with realistic head models from such simulations, we hope such approaches can help shed light on the neural origin of the EEG in the comming years.

Note that while we here used LFPy 2.0 \citep{HAGEN2018, HAGEN2019}, a python interface to NEURON \citep{CARNEVALE2006}, the calculation of the current dipole moments can easily be implemented into any framework where the transmembrane currents are available through the simple formula in given in eq.~(\ref{eq:dip_sum_trans}). Calculations of current dipole moments from morphologically complex cell models have been pursued before, for example to study the contriburion of spikes from single cells to EEG and MEG \citep{Murakami2006}, or to study the effect of the synaptic input location \citep{LINDEN2010, AHLFORS2015}.

%\subsection*{Comparisson to similar work}
Important work in interpreting EEGs in terms of the underlying neural activity has been done previously through use of "minimally sufficient" biophysical models, see for example \cite{Murakami2002, Murakami2003, Jones2007, Jones2009, Sliva2018, NEYMOTIN2020}. Here, "minimally sufficient" means that the cell models only had the minimum needed spatial structure (point neuron can not produce current dipole moments), only considered a few cell types, and employed simple synaptic connection rules. In particular, the Human Neocortical Neurosolver (HNN) \citep{NEYMOTIN2020} is designed to allow researchers to link measured EEG or MEG recordings to neural activity simulated through a pre-defined canonical neocortical column template network. HNN comes with an interactive GUI designed for users with little or no experience in computational modeling, and might therefore be an appropriate choice for researchers seeking to gain a better understanding of their EEG/MEG data. 
However, while the use of such minimally sufficient models allows for quick and direct comparison between simulated and recorded EEG signals, it is not (presently) compatible with simulating EEG or MEGs from biophysically detailed single cell- or network models, constructed from detailed experimental data \citep{Reimann2013, Egger2014, MARKRAM2015, HAGEN2016, Gratiy2018, Arkhipov2018, BILLEH2019}.  

%"HNN’s interactive GUI is designed for users with no formal computational neural modeling or software development experience to be able to develop and test hypotheses on the cellular- and circuit-level generators of their human data"
%"HNN’s model was chosen to be minimally sufficient to accurately account for the biophysical origin of the primary cur- rents that underlie EEG/MEG signals in a single brain area"
%"HNN’s default canonical neocortical column template network"
%"The number of cells in the network is adjustable in the Local Network Parameters window via the Cells tab, while maintaining at 3-to-1 pyramidal to inhibitory interneuron ratio in each layer. The con-nectivity pattern is fixed, but the synaptic weights between cell types can be adjusted in the Local Network menu and the Synaptic Gains menu."

A more high-level approach to simulate MEG/EEG signals from the underlying neural activity has been pursued through neural field or neural mass models \citep{Jirsa2002, David2003, Coombes2006, Deco2008, Bojak2010, Ritter2013}, which aims to model the evolution of coarse-grained variables such as the mean membrane potential or the firing rate of populations of neurons. Such coarse-graining drastically reduces the number of parameters and the computational burden of the simulation, and can be used to study the interplay among entire brain regions, and indeed run whole-brain simulations. 
The Virtual Brain (TVB) is an excellent example of a software for whole-brain network simulations \citep{TVB, Sanz-Leon2015, Ritter2013}, where detailed and potentially personalized head models can be used and combined with the connectivity between brain regions as identified by tractography based methods \citep{TVB}. To calculate measurement modalities like MEG and/or EEG signals from neural field or neural mass models, it is typically assumed that the population current dipole moments are roughly proportional to, for example, the average exitatory membrane potential \citep{Bojak2010, Ritter2013}, and the resulting current dipole moments can be used in combination with head models as presented in this paper, or through other softwares or techniques \citep{Gramfort2014} to calculate EEG signals.

Note that this suggests an intriguing future development, where one could, with high biological realism, calculate  the average post-synaptic current dipole moment of a given neural population, in practice done through simultaneously activating all outgoing synapses from a simulated population, and recording the total current dipole moment of the response (the kernel) \citep{HAGEN2016}. Convolving this kernel with the population firing rate would then provide a good approximation to the current dipole moment caused by the given neural population, and the EEG contribution could subsequently be calculated. This so-called kernel method has previously successfully been applied to the LFP \citep{HAGEN2016, Skaar2020, Telenczuk2020}, and it could provide a way to substantially increase the accuracy of LFP, EEG and MEG predictions from high-level simulations of neural activity.  

%Our approach differs in that it can be used with (pre-existing) biophysically detailed neurons and neural populations, and that it is very flexible and open. 
%Our approach can be used to calculate mappings from firing rates to EEGs through the kernel method, thereby meeting TVB.

%The Virtual Brain (TVB) is a software for full brain network simulations using detailed connectivity between brain regions as identified by tractography based methods \citep{TVB}, and can therefore use personalized head models. To calculate EEG signals TVB inherently assumes that the current dipole moment of neural populations is roughly proportional to 

%\cite{TVB, Sanz-Leon2015, Jirsa2002, Bojak2010, Ritter2013}

%"can be combined with forward models that allow the simulation of neuroimaging modalities (e.g., fMRI and EEG)."

%"Dipole strength D is thus roughly proportional to the average excitatory membrane potential"

%"For EEG predictions, volume conduction models for the skull and scalp surfaces are incorporated. Here it is assumed that the electric source activity can be well approximated by the fluctuation of equivalent current dipoles generated by excitatory neurons that have dendritic trees oriented roughly perpendicular to the cortical surface and that constitute the majority of neuronal cells (85 percent of all neurons). We neglect dipole contributions from inhibitory neurons, since they are only present in a low number (15 percent), and their dendrites fan out spherically. Therefore, dipole strength can be assumed to be roughly proportional to the average membrane potential of the excitatory population (Bojak et al., 2010)."

%"It is important to mention the similarity to neural mass models, which neglect spatial extensions but involve the same underlying assumptions on population coding and time-graining. "

%"Neural field equations are tissue level models that describe the spatiotemporal evolution of coarse grained variables such as synaptic or firing rate activity in populations of neurons. "

%"TVB allows the reproduction and evaluation of personalized configurations of the brain by using individual subject data."

%"a biologically realistic, large-scale connectivity of brain regions in the primate brain. Connectivity is mediated by long-range neural fiber tracts as identified by tractography based methods "

%"using neural mass models, building"
%"Neural field models have been developed over many years for their ability to capture the collective dynamics of relatively large areas of the brain in both analytically and computationally tractable forms (Beurle,

%Effectively neural field equations are tissue level models that describe the spatiotemporal evolution of coarse grained variables such as synaptic voltage or firing rate activity in populations of neurons. Some

%The lumped representation of the dynamics of a set of similar neurons via a common variable (e.g., mean firing rate and mean postsynaptic potential) is known as neural mass modeling

%Both neural field and neural mass modeling approaches embody the concept from statistical physics that macroscopic physical systems obey laws that are independent of the details of themicroscopic constituents of which they are built.

%"In TVB, our main interest lies in using the mesoscopic laws governing the behavior of neural populations and uncovering the laws driving the processes on the macroscopic brain network scale. The biophysical mechanisms available to microscopic single neuron approaches are absorbed in the mean field parameters on the mesoscopic scale and are not available for exploration other than through variation of the mean field parameters themselves. As a consequence, TVB represents a neuroinformatics tool that is designed to aid in the exploration of large-scale network mechanisms of brain functioning [see Ritter et al. (2013) for an example of modeling with TVB]."

%OpenMEEG (Gramfort et al., 2010) was used to generate the demonstration projectionmatrix, also known as lead-field or gain matrix, that corresponds to the EEG/MEEG forward solution."

%"TVB thus represents a unique tool to systematically investigate the dynamics of the brain, emphasizing its large-scale network nature and moving away from the study of isolated regional responses, thereby considering the function of each region in terms of the interplay among brain regions."

%Other approaches to simulate EEG signals have been
%Compare to other projects like  and TVB \citep{TVB}
%Cite others? Freesurfer? Sim4Life? Don Hagler? 

%TVB and HNN are doing important and related work, but at somewhat different levels

%\subsection*{Outlook}
While there are many examples of where detailed biophysical modelling of neural activity has helped interpretation of measured intracranial extracellular potentials in lab animals \citep{Einevoll2007, Blomquist2009, McColgan2017, Luo2018, Chatzikalymniou2018, Telenczuk2019}, much less has been done for human EEG signals, because studies of the healty human brain is necessarily limited to non-invasive technologies \citep{SILVA2013, Uhlirova2016, COHEN2017}.
However, given all the valuable insights that could be gained from an increased understanding of non-invasive measurements of neural activity in humans, an important challenge in modern neuroscience is to build on the mechanistic insights from such animal studies and use them for understanding non-invasive signals in humans \citep{SILVA2013, Uhlirova2016, COHEN2017, EINEVOLL2019, MAKI2019}.
The approach for calculating EEG/MEG signals in this paper should therefore ideally be used in combination with animal studies simultaneously measuring multisite laminar LFP (and MUA) signals within cortex, as well as EEG signals (see for example \cite{BRUYNS2017}) \citep{COHEN2017}.

%\subsection*{Nice final statement}
A better understanding of EEG signals could lead to important discoveries about how the brain works \citep{SILVA2013, Uhlirova2016, Pesaran2018, Ilmoniemi2019}, and provide new insights into mental disorders \citep{MAKI2019, Sahin2019}.
We believe that the work presented in this paper is an important step towards a better understanding of the EEG signal, which can potentially help us to take full advantage of this important brain signal in the future.

%An often quoted obstacle hindering a better understanding of the EEG signal in terms of the underlying neural sources is the sheer number of neurons contributing to the signal, which is typically expected to be from a few hundreds of thousands to millions of neurons.
%While this is indeed a daunting challenge, the EEG is still a relatively simple signal, with basically just four main sources: apical excitation/inhibition and basal excitation/inhibition to pyramidal cells. Connectivity is to a large degree fixed, meaning that we could probably map population firing rate to EEG.



%\begin{itemize}
%		\item Because of linearity, the EEG from a neural population is the sum of all the single-cell EEG contributions. However, if the spatial extent of the neural population (~ 1 mm) is small relative to the distance to the EEG electrodes (~ 1 cm), then individual current-dipoles can be summed, e.g., for each neural sub-population, or for the entire neural population, and the EEG signal can be computed based on the summed current dipoles. This could be a very promising approach for decomposing and understanding the EEG in large-scale neural  simulations like the cortical column simulations of the HBP, the Potians-Diesman  simulations of Hagen, or the Allen brain initiative.
%		\item Which populations are important? We found that the EEG calculated as the sum of the single-cell EEG contributions from the current dipoles of all ~80,000 cells, spatially distributed within a cylinder (panel B and E) was indistinguishable from the EEG calculated as the sum of the EEG from 4 population current dipoles in the center of the population. Each of these 4 current dipoles was the sum of the z-components of the current dipoles of the pyramidal cells in the different layers (panel F, L23E, L4E, L5E, L6E). We found that all pyramidal cell populations had a substantial EEG contribution, however, the largest was the L2/3 population. It should however be noted that this is not a very realistic scenario, because of current-based synapses etc.
%\end{itemize}


% \appendix
% \section{Point-Source vs. Line-Source Dipole Moment} \label{sec:point_line_cdm}
% \tvnnote{I don't think we need to go into this.}
% Considering the single-dipole neuron model, the line-source approximation gives a more accurate estimate than the point-source approximation. This section goes into the question of whether the choice of point-source or line-source approximation has an effect on the calculation of current dipole moments.
% 
% Transmembrane currents can be expressed as the spatial integral over the linear current density $i$. Following, the current dipole moment equation from transmembrane currents \eqref{eq:ptrans} is split into x-, y- and z- components, so that $\mathbf{p}(t) = p_x(t)\mathbf{\hat{x}} + p_y(t)\mathbf{\hat{y}} + p_z(t)\mathbf{\hat{z}}$, and each direction component is written as a function of $i$:
% 
% \begin{align}\label{eq:pxpypz}
% p_x(t) &= \sum_{n=1}^N\int x_n i_n(x,t) dx, \nonumber\\
% p_y(t) &= \sum_{n=1}^N\int y_n i_n(y,t) dy, \\
% p_z(t) &= \sum_{n=1}^N\int z_n i_n(z,t) dz, \nonumber
% \end{align}
% where $N$ is the total number of compartments.
% 
% Next, an example for applying the point-source approximation to calculate the current dipole moment from a dendritic stick is outlined. We assume a straight multicompartmental dendrite model with $N$ compartments, each of length $\Delta L$, elongated in the z-direction only. Its linear current density for the point-source approximation can be expressed as follows
% 
% \begin{equation}
% i_n(z,t) = I_n(t) \delta(z - z_n),
% \end{equation}
% where $I_n(t)$ is the space-independent current component and $z_n$ is the middle position of compartment $n$, i.e., where current can leave or enter. Plugging this into Equation \eqref{eq:pxpypz}, and integrating over the length of each compartment, $\Delta L$, the following expression for $p_z$ appears:
% 
% \begin{equation}
% p_z = \sum_{n=1}^N \int_{z_n - \frac{\Delta{L}}{2}}^{z_n + \frac{\Delta L}{2}} z_n I_n(t) \delta(z - z_n)dz = \sum_{n=1}^N z_n I_n(t).
% \end{equation}
% 
% When calculating the current dipole moment using the line-source approximation, the linear current density takes the following form:
% \begin{equation}
% i_n(z,t) = \frac{I_n(t)}{\Delta L}.
% \end{equation}
% Inserting this into Equation \eqref{eq:pxpypz} gives
% 
% \begin{align}
% p_z
% & = \sum_{n=1}^N \int_{z_n - \frac{\Delta L}{2}}^{z_n + \frac{\Delta L}{2}} z \frac{I_n(t)}{\Delta L}dz \nonumber \\
% & = \sum_{n=1}^N \frac{I_n(t)}{\Delta L} \big[\frac{1}{2} z^2\big]_{z_n - \frac{\Delta L}{2}}^{z_n + \frac{\Delta L}{2}} \\
% & = \sum_{n=1}^N z_n I_n(t). \nonumber
% \end{align}
% 
% Hence, we have found that the point-source and the line-source approximations will give the exact same results when calculating current dipole moments. The simpler point-source approximation is therefore preferable.
\section*{References}
\footnotesize
\bibliography{eeg_main}
\end{document}